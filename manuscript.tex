% Options for packages loaded elsewhere
% Options for packages loaded elsewhere
\PassOptionsToPackage{unicode}{hyperref}
\PassOptionsToPackage{hyphens}{url}
\PassOptionsToPackage{dvipsnames,svgnames,x11names}{xcolor}
%
\documentclass[
  letterpaper,
  DIV=11,
  numbers=noendperiod]{scrartcl}
\usepackage{xcolor}
\usepackage{amsmath,amssymb}
\setcounter{secnumdepth}{-\maxdimen} % remove section numbering
\usepackage{iftex}
\ifPDFTeX
  \usepackage[T1]{fontenc}
  \usepackage[utf8]{inputenc}
  \usepackage{textcomp} % provide euro and other symbols
\else % if luatex or xetex
  \usepackage{unicode-math} % this also loads fontspec
  \defaultfontfeatures{Scale=MatchLowercase}
  \defaultfontfeatures[\rmfamily]{Ligatures=TeX,Scale=1}
\fi
\usepackage{lmodern}
\ifPDFTeX\else
  % xetex/luatex font selection
\fi
% Use upquote if available, for straight quotes in verbatim environments
\IfFileExists{upquote.sty}{\usepackage{upquote}}{}
\IfFileExists{microtype.sty}{% use microtype if available
  \usepackage[]{microtype}
  \UseMicrotypeSet[protrusion]{basicmath} % disable protrusion for tt fonts
}{}
\makeatletter
\@ifundefined{KOMAClassName}{% if non-KOMA class
  \IfFileExists{parskip.sty}{%
    \usepackage{parskip}
  }{% else
    \setlength{\parindent}{0pt}
    \setlength{\parskip}{6pt plus 2pt minus 1pt}}
}{% if KOMA class
  \KOMAoptions{parskip=half}}
\makeatother
% Make \paragraph and \subparagraph free-standing
\makeatletter
\ifx\paragraph\undefined\else
  \let\oldparagraph\paragraph
  \renewcommand{\paragraph}{
    \@ifstar
      \xxxParagraphStar
      \xxxParagraphNoStar
  }
  \newcommand{\xxxParagraphStar}[1]{\oldparagraph*{#1}\mbox{}}
  \newcommand{\xxxParagraphNoStar}[1]{\oldparagraph{#1}\mbox{}}
\fi
\ifx\subparagraph\undefined\else
  \let\oldsubparagraph\subparagraph
  \renewcommand{\subparagraph}{
    \@ifstar
      \xxxSubParagraphStar
      \xxxSubParagraphNoStar
  }
  \newcommand{\xxxSubParagraphStar}[1]{\oldsubparagraph*{#1}\mbox{}}
  \newcommand{\xxxSubParagraphNoStar}[1]{\oldsubparagraph{#1}\mbox{}}
\fi
\makeatother

\usepackage{color}
\usepackage{fancyvrb}
\newcommand{\VerbBar}{|}
\newcommand{\VERB}{\Verb[commandchars=\\\{\}]}
\DefineVerbatimEnvironment{Highlighting}{Verbatim}{commandchars=\\\{\}}
% Add ',fontsize=\small' for more characters per line
\usepackage{framed}
\definecolor{shadecolor}{RGB}{241,243,245}
\newenvironment{Shaded}{\begin{snugshade}}{\end{snugshade}}
\newcommand{\AlertTok}[1]{\textcolor[rgb]{0.68,0.00,0.00}{#1}}
\newcommand{\AnnotationTok}[1]{\textcolor[rgb]{0.37,0.37,0.37}{#1}}
\newcommand{\AttributeTok}[1]{\textcolor[rgb]{0.40,0.45,0.13}{#1}}
\newcommand{\BaseNTok}[1]{\textcolor[rgb]{0.68,0.00,0.00}{#1}}
\newcommand{\BuiltInTok}[1]{\textcolor[rgb]{0.00,0.23,0.31}{#1}}
\newcommand{\CharTok}[1]{\textcolor[rgb]{0.13,0.47,0.30}{#1}}
\newcommand{\CommentTok}[1]{\textcolor[rgb]{0.37,0.37,0.37}{#1}}
\newcommand{\CommentVarTok}[1]{\textcolor[rgb]{0.37,0.37,0.37}{\textit{#1}}}
\newcommand{\ConstantTok}[1]{\textcolor[rgb]{0.56,0.35,0.01}{#1}}
\newcommand{\ControlFlowTok}[1]{\textcolor[rgb]{0.00,0.23,0.31}{\textbf{#1}}}
\newcommand{\DataTypeTok}[1]{\textcolor[rgb]{0.68,0.00,0.00}{#1}}
\newcommand{\DecValTok}[1]{\textcolor[rgb]{0.68,0.00,0.00}{#1}}
\newcommand{\DocumentationTok}[1]{\textcolor[rgb]{0.37,0.37,0.37}{\textit{#1}}}
\newcommand{\ErrorTok}[1]{\textcolor[rgb]{0.68,0.00,0.00}{#1}}
\newcommand{\ExtensionTok}[1]{\textcolor[rgb]{0.00,0.23,0.31}{#1}}
\newcommand{\FloatTok}[1]{\textcolor[rgb]{0.68,0.00,0.00}{#1}}
\newcommand{\FunctionTok}[1]{\textcolor[rgb]{0.28,0.35,0.67}{#1}}
\newcommand{\ImportTok}[1]{\textcolor[rgb]{0.00,0.46,0.62}{#1}}
\newcommand{\InformationTok}[1]{\textcolor[rgb]{0.37,0.37,0.37}{#1}}
\newcommand{\KeywordTok}[1]{\textcolor[rgb]{0.00,0.23,0.31}{\textbf{#1}}}
\newcommand{\NormalTok}[1]{\textcolor[rgb]{0.00,0.23,0.31}{#1}}
\newcommand{\OperatorTok}[1]{\textcolor[rgb]{0.37,0.37,0.37}{#1}}
\newcommand{\OtherTok}[1]{\textcolor[rgb]{0.00,0.23,0.31}{#1}}
\newcommand{\PreprocessorTok}[1]{\textcolor[rgb]{0.68,0.00,0.00}{#1}}
\newcommand{\RegionMarkerTok}[1]{\textcolor[rgb]{0.00,0.23,0.31}{#1}}
\newcommand{\SpecialCharTok}[1]{\textcolor[rgb]{0.37,0.37,0.37}{#1}}
\newcommand{\SpecialStringTok}[1]{\textcolor[rgb]{0.13,0.47,0.30}{#1}}
\newcommand{\StringTok}[1]{\textcolor[rgb]{0.13,0.47,0.30}{#1}}
\newcommand{\VariableTok}[1]{\textcolor[rgb]{0.07,0.07,0.07}{#1}}
\newcommand{\VerbatimStringTok}[1]{\textcolor[rgb]{0.13,0.47,0.30}{#1}}
\newcommand{\WarningTok}[1]{\textcolor[rgb]{0.37,0.37,0.37}{\textit{#1}}}

\usepackage{longtable,booktabs,array}
\usepackage{calc} % for calculating minipage widths
% Correct order of tables after \paragraph or \subparagraph
\usepackage{etoolbox}
\makeatletter
\patchcmd\longtable{\par}{\if@noskipsec\mbox{}\fi\par}{}{}
\makeatother
% Allow footnotes in longtable head/foot
\IfFileExists{footnotehyper.sty}{\usepackage{footnotehyper}}{\usepackage{footnote}}
\makesavenoteenv{longtable}
\usepackage{graphicx}
\makeatletter
\newsavebox\pandoc@box
\newcommand*\pandocbounded[1]{% scales image to fit in text height/width
  \sbox\pandoc@box{#1}%
  \Gscale@div\@tempa{\textheight}{\dimexpr\ht\pandoc@box+\dp\pandoc@box\relax}%
  \Gscale@div\@tempb{\linewidth}{\wd\pandoc@box}%
  \ifdim\@tempb\p@<\@tempa\p@\let\@tempa\@tempb\fi% select the smaller of both
  \ifdim\@tempa\p@<\p@\scalebox{\@tempa}{\usebox\pandoc@box}%
  \else\usebox{\pandoc@box}%
  \fi%
}
% Set default figure placement to htbp
\def\fps@figure{htbp}
\makeatother


% definitions for citeproc citations
\NewDocumentCommand\citeproctext{}{}
\NewDocumentCommand\citeproc{mm}{%
  \begingroup\def\citeproctext{#2}\cite{#1}\endgroup}
\makeatletter
 % allow citations to break across lines
 \let\@cite@ofmt\@firstofone
 % avoid brackets around text for \cite:
 \def\@biblabel#1{}
 \def\@cite#1#2{{#1\if@tempswa , #2\fi}}
\makeatother
\newlength{\cslhangindent}
\setlength{\cslhangindent}{1.5em}
\newlength{\csllabelwidth}
\setlength{\csllabelwidth}{3em}
\newenvironment{CSLReferences}[2] % #1 hanging-indent, #2 entry-spacing
 {\begin{list}{}{%
  \setlength{\itemindent}{0pt}
  \setlength{\leftmargin}{0pt}
  \setlength{\parsep}{0pt}
  % turn on hanging indent if param 1 is 1
  \ifodd #1
   \setlength{\leftmargin}{\cslhangindent}
   \setlength{\itemindent}{-1\cslhangindent}
  \fi
  % set entry spacing
  \setlength{\itemsep}{#2\baselineskip}}}
 {\end{list}}
\usepackage{calc}
\newcommand{\CSLBlock}[1]{\hfill\break\parbox[t]{\linewidth}{\strut\ignorespaces#1\strut}}
\newcommand{\CSLLeftMargin}[1]{\parbox[t]{\csllabelwidth}{\strut#1\strut}}
\newcommand{\CSLRightInline}[1]{\parbox[t]{\linewidth - \csllabelwidth}{\strut#1\strut}}
\newcommand{\CSLIndent}[1]{\hspace{\cslhangindent}#1}



\setlength{\emergencystretch}{3em} % prevent overfull lines

\providecommand{\tightlist}{%
  \setlength{\itemsep}{0pt}\setlength{\parskip}{0pt}}



 


\KOMAoption{captions}{tableheading}
\makeatletter
\@ifpackageloaded{caption}{}{\usepackage{caption}}
\AtBeginDocument{%
\ifdefined\contentsname
  \renewcommand*\contentsname{Table of contents}
\else
  \newcommand\contentsname{Table of contents}
\fi
\ifdefined\listfigurename
  \renewcommand*\listfigurename{List of Figures}
\else
  \newcommand\listfigurename{List of Figures}
\fi
\ifdefined\listtablename
  \renewcommand*\listtablename{List of Tables}
\else
  \newcommand\listtablename{List of Tables}
\fi
\ifdefined\figurename
  \renewcommand*\figurename{Figure}
\else
  \newcommand\figurename{Figure}
\fi
\ifdefined\tablename
  \renewcommand*\tablename{Table}
\else
  \newcommand\tablename{Table}
\fi
}
\@ifpackageloaded{float}{}{\usepackage{float}}
\floatstyle{ruled}
\@ifundefined{c@chapter}{\newfloat{codelisting}{h}{lop}}{\newfloat{codelisting}{h}{lop}[chapter]}
\floatname{codelisting}{Listing}
\newcommand*\listoflistings{\listof{codelisting}{List of Listings}}
\makeatother
\makeatletter
\makeatother
\makeatletter
\@ifpackageloaded{caption}{}{\usepackage{caption}}
\@ifpackageloaded{subcaption}{}{\usepackage{subcaption}}
\makeatother
\usepackage{bookmark}
\IfFileExists{xurl.sty}{\usepackage{xurl}}{} % add URL line breaks if available
\urlstyle{same}
\hypersetup{
  pdftitle={Manuscript},
  colorlinks=true,
  linkcolor={blue},
  filecolor={Maroon},
  citecolor={Blue},
  urlcolor={Blue},
  pdfcreator={LaTeX via pandoc}}


\title{Manuscript}
\author{}
\date{}
\begin{document}
\maketitle


\section{1. Introduction}\label{introduction}

Disposition:

\begin{itemize}
\tightlist
\item
  Recent focus on reproducibility and transparency
\item
  Implications for scientific workflow, collaboration, transparency\\
\item
  Lack of guidelines and examples of scientific workflows collaborative
  infrastructure which ensures this
\item
  Aim of this assignment\ldots{}
\end{itemize}

\textcolor{red}{The text below is a first draft of the introduciotn. The text itself needs alot of work to convey the scientific aims and background more understanble and precise. However, i feel that my overreaching "red line" and idea for the introducion is covered in this draft, and I want your thoughts in the logic of the text before revisiting it further. In addition, References are kept outside of the document due to possible conflicts when merging branches.}

Computational reproducibility and transparency are core principles of
scientific research, essential for ensuring the validity, reliability,
and credibility of scientific results. These principles facilitate
independent verification, foster critical evaluation, and enable the
accumulation of robust scientific knowledge. By adhering to these
principles, researchers enhance methodological rigor, promote
accountability, and strengthen trust in scientific results. As such,
computational reproducibility and transparency are not only key for
scientific progress but also represent ethical imperatives in the
advancement of scientific understanding. However, researchers often face
challenges in achieving this due to the requirement for specialized
computational frameworks and the absence of guidelines for scientific
workflows that facilitate these practices.

Achieving computational reproducibility and transparency is further
complicated when researchers collaborate on scientific projects due to
the complexity and often unstructured nature of scientific workflow. For
example, multiple versions of files may coexist, with changes made in
parallel, increasing the risk of inconsistencies and loss of critical
information. Although a more structured approach to tasks such as
project planning, data analysis, and scientific writing could
significantly improve collaboration, the lack of standardized workflows
and shared frameworks often hinders the implementation of such
practices.

To address the challenges of collaboration in scientific projects, this
assignment highlights tools and workflows designed to enhance
organizational efficiency and ensure reproducibility. This assignment
aims to (1) describe key components of robust, reproducible, and
transparent scientific workflows, including setting up a collaborative
work space, conduction simulations prior to data acquisition, including
data packages for distribution, and creating visualizations from data
packages, and (2) present collaborative framework for managing joint
research and writing efforts, with a focus on version control systems,
R, and GitHub, to streamline the creation of a transparent and
reproducible framework for scientific workflows. While the primary
outcome of this assignment is a PDF report, the accompanying GitHub
repository provides a comprehensive view of the collaborative workflow
and serves as an integral part of the project.

\section{2. Setting Up a Collaborative
Workspace}\label{setting-up-a-collaborative-workspace}

\begin{itemize}
\tightlist
\item
  How to create a repository in Git Hub
\item
  How to link it to Rstudio
\item
  How to create a project in your local computer and export it to GitHub
\item
  Basic commands for Git and GitHub fromt he RStudio terminal In this
  tutorial we assume that you will be working with RStudio and have
  already downloaded R and Rstudio. In addition you are going to need to
  have the version control software Git installed and have an account in
  GitHub. If you need guidance for this you can find a helpful tutorial
  \href{1\%20Setting\%20up\%20your\%20software\%20environment\%20–\%20A\%20Crash\%20Course\%20in\%20R}{here}.
\end{itemize}

\subsubsection{What is a repository?}\label{what-is-a-repository}

A repository is basically like a project box where you collect all the
files, data, graphs and code scripts from your project.\\
Online repositories can be accessed from the internet and from any
computer, while a local repository is only stored in a specific computer
and cannot be accessed elsewhere. When setting up a collaborative work
space its advantageous to have an online repository so that multiple
people can contribute from their own computer to a shared repository,
without having to send files by mail etc. In addition we can connect the
online repository what a local repository which allows us to work and
make changes using our own computer and then we can upload it to the
online repository.

\subsubsection{What is Git?}\label{what-is-git}

Git is a version control software that allows you to track the different
version of your files. It basically allows you to keep a detailed
history of changes you have done in your document and also what other
people have added or removed in your collaborative documents. Having a
version controls software set up for your workflow is very handy as it
prevents major losses of documents and changes, and if any error is
introduced in a document or code, you can track it back to see what and
who submitted it. This fosters reproducibility, transparency,
collaboration and robustness for your project.

\subsubsection{What is GitHub?}\label{what-is-github}

GitHub is a collaborative online platform that allows you to host and
join online repositories. its kinda like facebook for coding. gitHub
allows us to share and collaborate with the people on the same code at
the same time. It can also be used to host webpages and other stuff.

In this toturial we will only work with the RStudio interface and the
online GitHub interface. however, if you want an expanded commandline
and interface for GitHub you can use GitHUb CLI and/or GitHub Desktop.
See toturials here:
\href{GitHub\%20CLI\%20quickstart\%20-\%20GitHub\%20Docs}{GitHub CLI}
and
\href{Getting\%20started\%20with\%20GitHub\%20Desktop\%20-\%20GitHub\%20Docs}{GitHub
Desktop}.

\subsection{2.1 How to create a project that is connected between
RStudio and
GitHub?}\label{how-to-create-a-project-that-is-connected-between-rstudio-and-github}

When creating a new project and you want to link your local project with
an online repository, you can go about it to ways basicly.\\
a) You can create the online repository and then clone it down to you
computer\\
b) You can create a local repository and then push it online to GitHub

We´ll go through both options here, starting with the online repository.

Image file path:

resources/images/

\newpage

\subsubsection{2.1.1 Starting with an online
repository}\label{starting-with-an-online-repository}

\paragraph{\texorpdfstring{Step 1. Create a new online repository in
GitHub
(\href{Creating\%20Your\%20First\%20GitHub\%20Repository\%20and\%20Pushing\%20Code\%20-\%20YouTube}{Video
tutorial})}{Step 1. Create a new online repository in GitHub (Video tutorial)}}\label{step-1.-create-a-new-online-repository-in-github-video-tutorial}

\begin{wrapfigure}{lt}{0.5\textwidth}
  \centering
  \includegraphics[width=0.38\textwidth]{resources/images/image_1.png}
  \caption{Creating a new online repository.}
  \vspace{-1.5cm} % Adjust vertical spacing below the figure
\end{wrapfigure}

Once you´ve logged into GitHub, navigate to the top right corner of your
page and find the + tab. Drop it down to reveal the ``New repository''
option. Click on it.

This will take you ta the repository creation page.\\
Here you give your repository a name, a description of what it will
entail and wherever it is public or not.

You also have the options of adding a README file and a .gitignore file
upon creation, but it is possible to create these after the repository
is made as well.

\subparagraph{README}\label{readme}

A README file is a descriptive file that should explain what the
project/repository is about, how it is organized and what the data in it
means etc. Any additional information you want people to know when using
your repository should go into the README.

\begin{wrapfigure}{r}{0.4\textwidth}
  \vspace{-1cm} % Adjust vertical spacing to remove extra white space
  \centering
  \includegraphics[width=0.50\textwidth]{resources/images/image_3.png}
  \caption{Repository setup-page.}
  \vspace{-2cm} % Adjust vertical spacing to remove extra white space
\end{wrapfigure}

\subparagraph{.gitignore}\label{gitignore}

The .gitignore file is an information file that tells Git what types of
files it should track, or specifically not track. This is useful when
you for example don't want to track the generated images or graphs from
your code, but just your code.

So, now that you have created your first online repository you want to
connect it to your local computer.\\
You can do this multiple ways, but in this tutorial we´ll use commands
in the terminal to initialize

\paragraph{\texorpdfstring{\textbf{Step 2: Copy the Repository
URL}}{Step 2: Copy the Repository URL}}\label{step-2-copy-the-repository-url}

\begin{enumerate}
\def\labelenumi{\arabic{enumi}.}
\item
  Go to your GitHub repository page.
\item
  Click the green \textbf{Code} button.
\item
  Copy the repository URL:

  \begin{itemize}
  \tightlist
  \item
    For example:
    \textbf{\texttt{https://github.com/yourusername/repositoryname.git}}
  \end{itemize}
\end{enumerate}

\paragraph{\texorpdfstring{\textbf{Step 3: Open RStudio and Clone the
Repository}}{Step 3: Open RStudio and Clone the Repository}}\label{step-3-open-rstudio-and-clone-the-repository}

\begin{enumerate}
\def\labelenumi{\arabic{enumi}.}
\item
  Open RStudio.
\item
  Go to \textbf{File} → \textbf{New Project} → \textbf{Version Control}
  → \textbf{Git}.
\item
  Paste the GitHub repository URL into the ``Repository URL'' field.
\item
  Choose a folder on your local computer where you want to clone the
  repository.
\item
  Click \textbf{Create Project}.when doing a commit on a file that has
  been staged, that version of the file goes into the version history.
  It is also tracked.
\end{enumerate}

And tada! You have now cloned the online repository to your computer!
Great job!

\paragraph{Step 4 (Optional): Configure Git in
RStudio}\label{step-4-optional-configure-git-in-rstudio}

If this is your first time using Git with RStudio, you'll need to
configure your Git credentials. Open the RStudio terminal
(\textbf{Tools}~→~\textbf{Terminal}) or navigate to the \textbf{terminal
tab} at the top right of the RStudio interface and run the following
commands:

\begin{Shaded}
\begin{Highlighting}[]
\FunctionTok{git}\NormalTok{ config –global user.name }\StringTok{"Your Name"}

\FunctionTok{git}\NormalTok{ config }\AttributeTok{{-}{-}global}\NormalTok{ user.email }\StringTok{"your\_email@example.com"}
\end{Highlighting}
\end{Shaded}

Replace the placeholder names in ``Your Name'' and
``your\_email@example.com'' with your own.

Also, if Git has not yet been initiated in your RStudio project you can
use the command:

\begin{Shaded}
\begin{Highlighting}[]
\FunctionTok{git}\NormalTok{ init}
\end{Highlighting}
\end{Shaded}

To check whet ever Git is initialized in your project you can write:

\begin{Shaded}
\begin{Highlighting}[]
\FunctionTok{git}\NormalTok{ status}
\end{Highlighting}
\end{Shaded}

If git is not initialized an error message will show up. Then just run
the \texttt{git\ init} command and follow the instructions.

\subsubsection{2.1.2 Starting with a local project in
RStudio}\label{starting-with-a-local-project-in-rstudio}

Now, what if you wanted to do it the other way around, like if you
already have a local project on your computer and want to create an
online repository for it?

\paragraph{\texorpdfstring{\textbf{Step 1. Create a New Local Project in
RStudio}}{Step 1. Create a New Local Project in RStudio}}\label{step-1.-create-a-new-local-project-in-rstudio}

If you already have a local project, you can skip this step. If not:

\begin{enumerate}
\def\labelenumi{\arabic{enumi}.}
\item
  Open \textbf{RStudio}.
\item
  Go to \textbf{File \textgreater{} New Project \textgreater{} New
  Directory \textgreater{} New Project}.
\item
  Choose a folder where you want the project to live and give it a name.
\item
  Make sure to check the box \textbf{Create a Git repository}.
\item
  Click \textbf{Create Project}.
\end{enumerate}

This initializes a local Git repository in your project directory.

If you already have made a project but it is not connected to a Git
repository you can do it like this:

\begin{enumerate}
\def\labelenumi{\arabic{enumi}.}
\tightlist
\item
  Navigate to \textbf{Tools \textgreater{} Project Options
  \textgreater{} Git/SVN}.
\item
  Select \textbf{Git} and click \textbf{Yes} when prompted to initialize
  a Git repository for your project.
\end{enumerate}

\paragraph{Step 2. Create a New Repository on
GitHub}\label{step-2.-create-a-new-repository-on-github}

\begin{enumerate}
\def\labelenumi{\arabic{enumi}.}
\item
  Create a new repository in Git Hub like previously in section 2.1.1
  step 1

  \begin{itemize}
  \tightlist
  \item
    Do \textbf{not} initialize the repository with a README,
    \textbf{\texttt{.gitignore}}, or license (we'll connect the existing
    local repository later).
  \end{itemize}
\end{enumerate}

You now have a new, empty GitHub repository.

\paragraph{Step 3. Link the Local project to the GitHub
Repository}\label{step-3.-link-the-local-project-to-the-github-repository}

\begin{enumerate}
\def\labelenumi{\arabic{enumi}.}
\tightlist
\item
  Copy the URL in the same way as previously in section 2.1.1 step 2.
\item
  Open the \textbf{Terminal} in RStudio (or use any terminal on your
  computer).
\item
  Navigate to your project folder, if you're not already there, by
  clicking on the dropdown menu in the top right corner of the RStudio
  interface and choose your project.
\item
  Add the GitHub repository as the remote origin using this command in
  the terminal:
\end{enumerate}

\begin{Shaded}
\begin{Highlighting}[]
\FunctionTok{git}\NormalTok{ remote add origin https://github.com/yourusername/your{-}repository.git}
\end{Highlighting}
\end{Shaded}

\begin{enumerate}
\def\labelenumi{\arabic{enumi}.}
\setcounter{enumi}{4}
\tightlist
\item
  Verify the remote connection using this command in the terminal
  \texttt{git\ remote\ -v}
\end{enumerate}

You should be able to see something like this:

\begin{Shaded}
\begin{Highlighting}[]
\ExtensionTok{origin}\NormalTok{  https://github.com/yourusername/your{-}repo.git }\ErrorTok{(}\ExtensionTok{fetch}\KeywordTok{)}
\ExtensionTok{origin}\NormalTok{  https://github.com/yourusername/your{-}repo.git }\ErrorTok{(}\ExtensionTok{push}\KeywordTok{)}
\end{Highlighting}
\end{Shaded}

That means you have now successfully established a connection between
your local project and the online repository on GitHub. Congratulations!

\subsection{2.2 Workflow between local and online
repositories}\label{workflow-between-local-and-online-repositories}

Now that we have connected our local repository with the online one, we
can start to pushing some code! But before we jump into the commands for
transferring files between our local and online repository, we should
better understand how these processes work.

Take a look at the figure below:

\begin{figure}[H]

{\centering \includegraphics[width=1\linewidth,height=0.6\textheight]{resources/images/Local_vs_online_workflow.png}

}

\caption{Visaluzation of local and online workflow in Rstudio.}

\end{figure}%

Your \textbf{working tree} (also called \textbf{working directory}) is
the project folder you are currently working in on your local computer.
This is where you do all your edits on your files and code.

The \textbf{index/staging area} is a list is a list of files that Git is
tracking for changes. When you ``stage'' files you tell Git to include
these files in the next commit.

The \textbf{local branch} is a version of your project that exists
entirely on your local computer. When you ``commit'' changes, you create
a permanent snapshot of the files currently staged in the index and save
them to your local repository.

The \textbf{remote tracking ref} is where you track the state of the
online repository. When you ``fetch'' changes from the online
repository, Git downloads the latest updates from the online repository
but without merging it into your working directory. This allows you to
see changes from collaborators or updates from the remote repository
while keeping your working tree unaffected until you explicitly merge or
rebase the changes.

\paragraph{2.2.1 Commands}\label{commands}

Okay, now we can take a look at the commands we can use for this
workflow.

Lets say you have edited some files in your working directory and want
to add them to your~\textbf{staging area}~before committing them to your
repository.~ If you only want to ``stage'' some files, but not all, you
can use the command in the terminal:

\begin{Shaded}
\begin{Highlighting}[]
\FunctionTok{git}\NormalTok{ add file\_name.txt}
\end{Highlighting}
\end{Shaded}

Just replace the file\_name.txt with the name of the file that you want
to stage. Remember to also include the file extentions (like .txt or
.pdf etc.)

If you want to stage \textbf{all files in your repository} that have had
changes to them you can write:

\begin{Shaded}
\begin{Highlighting}[]
\FunctionTok{git}\NormalTok{ add }\AttributeTok{{-}A}
\end{Highlighting}
\end{Shaded}

This command stages \textbf{all changes} in the repository, including
modified files, newly created files and deleted files.

Great! Now your edited files are ready to be committed! You can commit
your files using this command:

\begin{Shaded}
\begin{Highlighting}[]
\FunctionTok{git}\NormalTok{ commit }\AttributeTok{{-}m} \StringTok{"Commit message"}
\end{Highlighting}
\end{Shaded}

When you run this commit command, all the files that you have staged are
committed to your \textbf{local branch!} Its good to include a
\textbf{descriptive commit message} that explains what your commit
entails. That way its easier to track where changes or errors are
introduced in your repository. Messages could be ``Changed font
headings'' or ``Added statistical ANOVA analysis to data processing''.

Now that you have committed the files you want to push them to your
remote directory using:

\begin{Shaded}
\begin{Highlighting}[]
\FunctionTok{git}\NormalTok{ push}
\end{Highlighting}
\end{Shaded}

This command uploads everything you have committed so far this session
to your online repository and updates it based on your commits. Using
this command you can push multiple commits at the same time.

Now lets say your colleague, who you are collaborate with, has added a
new file to the online repository on GitHub. You want to pull that
document from the online directory down to your local repositopry and
start editing it. The most straightforward way is to use:

\begin{Shaded}
\begin{Highlighting}[]
\FunctionTok{git}\NormalTok{ pull}
\end{Highlighting}
\end{Shaded}

This command pulls the latest changes from your remote repository and
merges them into your~\textbf{current branch}~in your local repository,
so that they are exactly the same. Then you can just start editing and
making changes.

If you don't want to fully copy the online repository yet, but take a
look at it first before merging it ito your working directory you can
use the command

\begin{Shaded}
\begin{Highlighting}[]
\FunctionTok{git}\NormalTok{ fetch}
\end{Highlighting}
\end{Shaded}

This command downloads the changes from the remote repository
(e.g.,~\textbf{\texttt{origin}}) and updates your remote-tracking
branches (e.g.,~\textbf{\texttt{origin/main}}) without modifying your
working directory or local branch.

To look at the branch you have fetched you can use one of these two
commands:

\begin{Shaded}
\begin{Highlighting}[]
\FunctionTok{git}\NormalTok{ log HEAD..origin/main}

\FunctionTok{git}\NormalTok{ diff HEAD..origin/main}
\end{Highlighting}
\end{Shaded}

\textbf{git log} will give you a list of the commits that are not in
your local branch (\textbf{HEAD}) together with their metadata (authors,
date etc.). This is helpful if you want to quickly answer ``What was
changed?'' and ``Who changed it?''.

\textbf{git diff} will show you the actual changes that are between two
points, ex. your local branch (\textbf{HEAD}) and your fetched branch
(\textbf{origin/main}). git diff will show you the specific lines of
code that were added modified or deleted in the fetched branch. This is
helpful if you want to know exactly what was changed.

If you then want to integrate the fetched branch with your current
branch, you can use the command:

\begin{Shaded}
\begin{Highlighting}[]
\FunctionTok{git}\NormalTok{ merge}
\end{Highlighting}
\end{Shaded}

Using this command, Git creates a \textbf{merge commit}, which combines
the changes from both branches while preserving the complete commit
history of both branches. The merge commit explicitly shows the point
where the branches diverged and were brought together.

An alternative way to integrate the fetched branch is to use:

\begin{Shaded}
\begin{Highlighting}[]
\FunctionTok{git}\NormalTok{ rebase}
\end{Highlighting}
\end{Shaded}

This command does~\textbf{not create a merge commit}. Instead,
it~\textbf{rewrites the commit history}~of your current branch by
replaying its commits on top of the fetched branch, creating
a~\textbf{linear history}. Essentially, it re-aligns your current branch
so that it starts from the latest commit of the fetched branch, as if
your changes were made after the fetched branch's changes. This results
in a cleaner, more linear commit history.

Questions:

\begin{itemize}
\item
  add sections about README files and .gitignore
\item
  should I add a section about the concole and the terminal andhow they
  are used? Or could this perhaps go into the Introduction?
\item
  Should we include stuff about creating branches?
\end{itemize}

\textbf{Disclaimer:} This section was written with the help of SIKT KI
chat using the gpt-4o model. The AI was used to verify code chunks,
summarize steps in an organized format and rewrite original text for
better grammar and flow. The author takes full responsibility for the
resulting output. \href{Sikt\%20KI}{Link to AI model}.

\newpage

\section{3. Conducting Simulations Before Data
Acquisition}\label{conducting-simulations-before-data-acquisition}

Conducting simulations prior to data acquisition is useful for several
reasons, including:

\begin{enumerate}
\def\labelenumi{\arabic{enumi}.}
\tightlist
\item
  \emph{Model design}. By sampling from the prior distributions, we can
  understand our expectations about potential outcomes. This pre-data
  exploration offers insights into the implications of the prior
  assumptions.
\item
  \emph{Model checking}. Once a model is updated using real data,
  simulating implied data can help assess the success of the fit and
  explore the model's behavior.
\item
  \emph{Software validation}. In order to ensure that our model fitting
  software is functioning correctly, it is helpful to simulate
  observations under a known model and then attempt to recover the
  parameter values from which the data were simulated.
\item
  \emph{Research design}. Simulating observations based on our
  hypothesis allows for an evaluation of the research design's
  effectiveness. This is similar to conducting a \emph{power analysis},
  but the possibilities are broader.\\
\item
  \emph{Forecasting}. Estimates derived from simulations can be utilized
  to generate predictions for new cases and future observations
  (\citeproc{ref-McElreath_2018}{\emph{Statistical Rethinking}, 2018}).
\end{enumerate}

To illustrate some of the properties of conducting simulations, we will
consider the following hypothesis: \textbf{Increasing daily exercise
time is associated with lower blood pressure levels in adults}.

To investigate this hypothesis, it is valuable to invest effort into
finding methods that distinguish causal inferences from associations. A
helpful first step to do this is to construct a causal model that is
separate from the statistical model. The simplest graphical
representation of such a causal model is a \emph{DIRECTED ACYCLIC
GRAPH}, commonly referred to as a \emph{DAG}
(\citeproc{ref-digitale_tutorial_2022}{Digitale et al., 2022};
\citeproc{ref-McElreath_2018}{\emph{Statistical Rethinking}, 2018}).

A DAG primarily serves to represent our prior knowledge about biological
and behavioral systems that may confound the specific causal research
question. In our example, we aim to investigate the causal effect of
increasing exercise time, denoted as ``E'', on blood pressure, ``B''. A
basic DAG for this relationship might look like this (Figure 1):

\begin{figure}[H]

{\centering \pandocbounded{\includegraphics[keepaspectratio]{manuscript_files/figure-pdf/Fig-dag1-1.pdf}}

}

\caption{A DAG, assuming that the causal effect of increasing exercise
time (E) on blood pressure (B) is not confounded.}

\end{figure}%

This DAG assumes that the causal effect of increasing exercise time on
blood pressure is not confounded, which is arguably a naive assumption.
For instance, both age and diet could potentially confound this
relationship. A more comprehensive DAG that accounts for these
confounders might appear as follows (Figure 2):

\begin{figure}[H]

{\centering \pandocbounded{\includegraphics[keepaspectratio]{manuscript_files/figure-pdf/Fig-dag2-1.pdf}}

}

\caption{A DAG, assuming that the causal effect of increasing exercise
time (E) on blood pressure (B) is confounded by both age (A) and diet
(D).}

\end{figure}%

In this adjusted DAG, we presume that age directly influences blood
pressure--- an assumption supported by the correlation between aging and
increased blood pressures (\citeproc{ref-Miall_1967}{Miall \& Lovell,
1967}). Additionally, age may indirectly influence exercise time since
older individuals often engage in less physical activity. Diet is also
included as a confounding factor affecting blood pressure, linked to
dietary habits like high-fat intake (\citeproc{ref-Wilde_2000}{Dw et
al., 2000}).

Given that we believe our DAG to be accurate, we can construct our model
with stronger scientific justifications regarding which confounding
factors should be accounted for and how we will do so.

\textbf{Model design}

Having established the confounding factors of age and diet in the
relationship between exercise time and blood pressure, we are now
prepared to design our study. We plan to recruit 100 sedentary adults
aged between 20 and 70 for a one-year training study, during which
participants will adhere to the WHO guideline of 150 minutes of
endurance exercise per week. Unfortunately, we will only be able to
measure blood pressure after the intervention period due to logistical
constraints. Therefore, we can only investigate whether there is an
association between how many minutes of physical activity the
participants complete and their blood pressure levels after the
intervention. To control for age and diet, we will document
participants' ages at the time of inclusion and assess their diets using
a binomial scale, where 1 indicates a healthy diet and 0 indicates an
unhealthy diet.

The model can now be described mathematically, for example in terms of a
simple linear regression model:

\begin{align}
\operatorname{BloodPressure}_i &= \beta_0 + \beta_1 \cdot \operatorname{Exercise}_i 
+ \beta_2 \cdot \operatorname{Age}_i + \beta_3 \cdot \operatorname{Diet}_i + \epsilon_i \\
\end{align}

Where:

\begin{align}
\beta_0 &: \text{Intercept}\\
\beta_1 &: \text{Coefficient for exercise time}\\
\beta_2 &: \text{Coefficient for age}\\
\beta_3 &: \text{Coefficient for diet}\\
\epsilon_i &: \text{Random error term}\\
\end{align}

The linear regression model can be implemented in R as follows:

\begin{Shaded}
\begin{Highlighting}[]
\CommentTok{\# Example dataset}
\NormalTok{data }\OtherTok{\textless{}{-}} \FunctionTok{data.frame}\NormalTok{(}
  \AttributeTok{exercise =} \FunctionTok{rnorm}\NormalTok{(}\DecValTok{100}\NormalTok{, }\AttributeTok{mean=}\DecValTok{150}\NormalTok{, }\AttributeTok{sd=}\DecValTok{20}\NormalTok{),}
  \AttributeTok{age =} \FunctionTok{rnorm}\NormalTok{(}\DecValTok{100}\NormalTok{, }\AttributeTok{mean=}\DecValTok{45}\NormalTok{, }\AttributeTok{sd=}\DecValTok{15}\NormalTok{),}
  \AttributeTok{diet =} \FunctionTok{rnorm}\NormalTok{(}\DecValTok{100}\NormalTok{, }\AttributeTok{mean=}\DecValTok{2}\NormalTok{, }\AttributeTok{sd=}\DecValTok{1}\NormalTok{),}
  \AttributeTok{bloodpressure =} \FunctionTok{rnorm}\NormalTok{(}\DecValTok{100}\NormalTok{, }\AttributeTok{mean=}\DecValTok{120}\NormalTok{, }\AttributeTok{sd=}\DecValTok{10}\NormalTok{)}
\NormalTok{)}

\CommentTok{\# Fit a simple linear regression model}
\NormalTok{model }\OtherTok{\textless{}{-}} \FunctionTok{lm}\NormalTok{(bloodpressure }\SpecialCharTok{\textasciitilde{}}\NormalTok{ exercise }\SpecialCharTok{+}\NormalTok{ age }\SpecialCharTok{+}\NormalTok{ diet, }\AttributeTok{data =}\NormalTok{ data)}

\CommentTok{\# Summarize the model results}
\FunctionTok{summary}\NormalTok{(model)}
\end{Highlighting}
\end{Shaded}

\textbf{Run simulations}

We will now run simulations to evaluate our model design. First, we need
to define our parameter values:

\begin{align}
\operatorname{BP}_i & = \beta_0 + \beta_E \cdot \operatorname{E}_i + \beta_A \cdot \operatorname{A}_i + \beta_D \cdot \operatorname{D}_i + \epsilon_i \\
\operatorname{E}_i & \sim \operatorname{Normal}(\mu_E, \sigma_E) \\
\operatorname{A}_i & \sim \operatorname{Normal}(\mu_A, \sigma_A) \\
\operatorname{D}_i & \sim \operatorname{Binomial}(n=1, p=0.5) \\
\epsilon_i & \sim \operatorname{Normal}(\mu_\epsilon, \sigma_\epsilon) \\
\beta_0 & = 120 \quad (\text{intercept}) \\
\beta_E & = -0.1 \quad (\text{beta for exercise}) \\
\beta_A & = 0.2 \quad (\text{beta for age}) \\
\beta_D & = -5 \quad (\text{beta for diet}) \\
\mu_E & = 150 \quad (\text{mean exercise}) \\
\mu_A & = 45 \quad (\text{mean age}) \\
\mu_\epsilon & = 0 \quad (\text{mean error term}) \\
\sigma_E & = 20 \quad (\text{exercise standard deviation}) \\
\sigma_A & = 10 \quad (\text{age standard deviation}) \\
\sigma_\epsilon & = 5 \quad (\text{error term standard deviation}) \\
\end{align}

After defining our parameter values, we can simulate data using these
parameters. To do this, it is convenient to create a function where you
specify information about each parameter that can be manipulated later,
thereby enabling experimentation with different values to assess the
model's robustness. Below is the code for this simulation:

\begin{Shaded}
\begin{Highlighting}[]
\CommentTok{\# Simulate data based on the hypothetical model}

\CommentTok{\# A basic function}
\NormalTok{sim.fun }\OtherTok{\textless{}{-}} \ControlFlowTok{function}\NormalTok{(}\AttributeTok{n =} \DecValTok{100}\NormalTok{,}
                \AttributeTok{intercept =} \DecValTok{120}\NormalTok{,}
                \AttributeTok{mean\_exercise =} \DecValTok{150}\NormalTok{,}
                \AttributeTok{beta\_exercise =} \SpecialCharTok{{-}}\FloatTok{0.1}\NormalTok{,}
                \AttributeTok{sigma\_exercise =} \DecValTok{20}\NormalTok{,}
                \AttributeTok{mean\_age =} \DecValTok{45}\NormalTok{,}
                \AttributeTok{beta\_age =} \FloatTok{0.2}\NormalTok{,}
                \AttributeTok{sigma\_age =} \DecValTok{10}\NormalTok{,}
                \AttributeTok{size\_diet =} \DecValTok{1}\NormalTok{,}
                \AttributeTok{prob\_diet =} \FloatTok{0.5}\NormalTok{,}
                \AttributeTok{beta\_diet =} \SpecialCharTok{{-}}\DecValTok{5}\NormalTok{, }
                \AttributeTok{mean\_epsilon =} \DecValTok{0}\NormalTok{,}
                \AttributeTok{sigma\_epsilon =} \DecValTok{5}\NormalTok{) \{}
  
  \CommentTok{\# This is the body }
\NormalTok{  epsilon }\OtherTok{\textless{}{-}} \FunctionTok{rnorm}\NormalTok{(n, }\AttributeTok{mean =}\NormalTok{ mean\_epsilon, }\AttributeTok{sd =}\NormalTok{ sigma\_epsilon) }\CommentTok{\#error term }
\NormalTok{  diet }\OtherTok{\textless{}{-}} \FunctionTok{rbinom}\NormalTok{(n, }\AttributeTok{size =}\NormalTok{ size\_diet, }\AttributeTok{prob =}\NormalTok{ prob\_diet) }\CommentTok{\#simulate diet }
\NormalTok{  age }\OtherTok{\textless{}{-}} \FunctionTok{rnorm}\NormalTok{(n, }\AttributeTok{mean =}\NormalTok{ mean\_age, }\AttributeTok{sd =}\NormalTok{ sigma\_age) }\CommentTok{\#simulate age}
\NormalTok{  exercise }\OtherTok{\textless{}{-}} \FunctionTok{rnorm}\NormalTok{(n, }\AttributeTok{mean =}\NormalTok{ mean\_exercise, }\AttributeTok{sd =}\NormalTok{ sigma\_exercise) }\CommentTok{\#simulate exercise duration}
\NormalTok{  bloodpressure }\OtherTok{\textless{}{-}}\NormalTok{ intercept }\SpecialCharTok{+}\NormalTok{ (beta\_exercise }\SpecialCharTok{*}\NormalTok{ exercise) }\SpecialCharTok{+}\NormalTok{ (beta\_age }\SpecialCharTok{*}\NormalTok{ age) }\SpecialCharTok{+} 
\NormalTok{    (beta\_diet }\SpecialCharTok{*}\NormalTok{ diet) }\SpecialCharTok{+}\NormalTok{ epsilon}
  
  
  
  \CommentTok{\# The output}
  \FunctionTok{return}\NormalTok{(}\FunctionTok{data.frame}\NormalTok{(age, exercise, diet, bloodpressure, epsilon))}
  
\NormalTok{\}}


\CommentTok{\# Store the data frame as "sim\_data"}
\NormalTok{sim\_data }\OtherTok{\textless{}{-}} \FunctionTok{sim.fun}\NormalTok{()}
\end{Highlighting}
\end{Shaded}

With our simulated dataset, we can now test our hypotheses using the
linear model: \emph{lm(bloodpressure \textasciitilde{} exercise + age +
diet, sim\_data)}. The output will indicate whether there is an observed
effect of exercise on blood pressure:

\begin{verbatim}

Call:
lm(formula = bloodpressure ~ exercise + age + diet, data = sim_data)

Residuals:
     Min       1Q   Median       3Q      Max 
-11.0726  -2.9827   0.1707   2.5103  11.6084 

Coefficients:
             Estimate Std. Error t value Pr(>|t|)    
(Intercept) 124.31364    4.19317  29.647  < 2e-16 ***
exercise     -0.11219    0.02347  -4.779 6.33e-06 ***
age           0.15756    0.04348   3.623 0.000467 ***
diet         -4.99045    0.93906  -5.314 6.94e-07 ***
---
Signif. codes:  0 '***' 0.001 '**' 0.01 '*' 0.05 '.' 0.1 ' ' 1

Residual standard error: 4.532 on 96 degrees of freedom
Multiple R-squared:  0.3928,    Adjusted R-squared:  0.3738 
F-statistic:  20.7 on 3 and 96 DF,  p-value: 2.001e-10
\end{verbatim}

In this model, we can see that each minute of increased exercise time is
associated with a \textbf{-0.112} mmHg lower blood pressure, when age
and diet are included as covariates.

Now it is time to play with our simulation! Lets say we only manage to
include 50 participents, what will happen whit our results?

This can easily be tested by manipulating our function and then running
the model, like this:

\begin{Shaded}
\begin{Highlighting}[]
\CommentTok{\# Changing number of observations to 50!}
\FunctionTok{set.seed}\NormalTok{(}\DecValTok{7}\NormalTok{)}
\NormalTok{sim\_data2 }\OtherTok{\textless{}{-}} \FunctionTok{sim.fun}\NormalTok{(}\AttributeTok{n =} \DecValTok{50}\NormalTok{)}
\end{Highlighting}
\end{Shaded}

\begin{verbatim}

Call:
lm(formula = bloodpressure ~ exercise + age + diet, data = sim_data2)

Residuals:
    Min      1Q  Median      3Q     Max 
-9.1410 -3.9726 -0.3834  3.5898 11.2395 

Coefficients:
             Estimate Std. Error t value Pr(>|t|)    
(Intercept) 111.46606    6.72708  16.570  < 2e-16 ***
exercise     -0.03144    0.03937  -0.799 0.428622    
age           0.19960    0.07677   2.600 0.012491 *  
diet         -6.03041    1.46590  -4.114 0.000159 ***
---
Signif. codes:  0 '***' 0.001 '**' 0.01 '*' 0.05 '.' 0.1 ' ' 1

Residual standard error: 5.024 on 46 degrees of freedom
Multiple R-squared:  0.3392,    Adjusted R-squared:  0.2961 
F-statistic: 7.872 on 3 and 46 DF,  p-value: 0.0002419
\end{verbatim}

The output for this model may reveal that the \textbf{-0.031} mmHg
effect of exercise is no longer significant, suggesting that 50
participants may not be sufficient to detect the true effect of exercise
which we know is \textbf{-0.1} mmHg. However, this is just one
simulation; results may vary due to random chance. It is therefore
prudent to conduct multiple simulations to evaluate the robustness of
our model, which can be executed in a loop as demonstrated below:

\begin{Shaded}
\begin{Highlighting}[]
\FunctionTok{set.seed}\NormalTok{(}\DecValTok{1}\NormalTok{)}
\NormalTok{results }\OtherTok{\textless{}{-}} \FunctionTok{list}\NormalTok{()}

\ControlFlowTok{for}\NormalTok{(i }\ControlFlowTok{in} \DecValTok{1}\SpecialCharTok{:}\DecValTok{1000}\NormalTok{) \{}
  
\NormalTok{  dat1 }\OtherTok{\textless{}{-}} \FunctionTok{sim.fun}\NormalTok{()}
\NormalTok{  dat2 }\OtherTok{\textless{}{-}} \FunctionTok{sim.fun}\NormalTok{(}\AttributeTok{n =} \DecValTok{50}\NormalTok{)}
  
  
\NormalTok{  m1 }\OtherTok{\textless{}{-}} \FunctionTok{lm}\NormalTok{(bloodpressure }\SpecialCharTok{\textasciitilde{}}\NormalTok{ exercise }\SpecialCharTok{+}\NormalTok{ age }\SpecialCharTok{+}\NormalTok{ diet, }\AttributeTok{data =}\NormalTok{ dat1)}
\NormalTok{  m2 }\OtherTok{\textless{}{-}} \FunctionTok{lm}\NormalTok{(bloodpressure }\SpecialCharTok{\textasciitilde{}}\NormalTok{ exercise }\SpecialCharTok{+}\NormalTok{ age }\SpecialCharTok{+}\NormalTok{ diet, }\AttributeTok{data =}\NormalTok{ dat2)}

  
\NormalTok{  results[[i]] }\OtherTok{\textless{}{-}} \FunctionTok{data.frame}\NormalTok{(}\AttributeTok{model =} \FunctionTok{c}\NormalTok{(}\StringTok{"m1"}\NormalTok{, }\StringTok{"m2"}\NormalTok{),}
             \AttributeTok{estimate =} \FunctionTok{c}\NormalTok{(}\FunctionTok{coef}\NormalTok{(m1)[}\DecValTok{2}\NormalTok{],}
                          \FunctionTok{coef}\NormalTok{(m2)[}\DecValTok{2}\NormalTok{]))}
  
  
  
\NormalTok{\}}
\end{Highlighting}
\end{Shaded}

To visualize the estimates of model 1 and model 2, it is convenient to
create a figure similar to Figure 3.

\begin{figure}[H]

{\centering \pandocbounded{\includegraphics[keepaspectratio]{manuscript_files/figure-pdf/Simulation figure-1.pdf}}

}

\caption{Demonstrating the variation in the estimated effect of exercise
on blood pressure using 100 participants (m1) compared to 50
participants (m2) by simulating 1,000 different datasets.}

\end{figure}%

In the simulated models shown in Figure 3, the estimate for model m1
clusters more tightly around -0.1, which represents the true effect of
exercise on blood pressure as specified. In contrast, model m2 exhibits
more variation, indicating that the effect is less reliably detected
with only 50 participants.

This brief introduction illustrates how to conduct basic data
simulations before data acquisition, empowering you to enhance the
robustness of your research!

\section{4. Including Data Packages for
Distribution}\label{including-data-packages-for-distribution}

\section{5. Creating Visualizations from Data
Packages}\label{creating-visualizations-from-data-packages}

As the title suggests, this section will aim to describe how to create
plots in different ways with a focus on reproducibility and
collaboration. In this perspective, I believe making descriptive
comments about you wrangle and plot your data is key, especially with
collaboration in mind - both for your own and others sake. Doing this
should make it much easier for yourself and others to pick up the work
where you left of.

\subsubsection{Prior to plotting}\label{prior-to-plotting}

Since we already have a data package that should cleanup and ready our
data (or at least close to ready) for visualization, we can simply
import the data from this package.

The package from which we will import data is organized by test, meaning
one xlsx file per test with no other info than what was measured in each
specific test. Thus, it is necessary to wrangle and tidy the data a bit
further, to get all the information we would like to visualize (such as
what age group, condition and allocation each participant had). An
almost surprising amount of time can go into this process before you
even start plotting, but tidy data will make plotting quite a lot
easier.

\begin{Shaded}
\begin{Highlighting}[]
\FunctionTok{library}\NormalTok{(tidyverse)}
\FunctionTok{library}\NormalTok{(dplyr)}

\DocumentationTok{\#\# Importing the data from the data package}

\NormalTok{humac.dat }\OtherTok{\textless{}{-}}\NormalTok{ reliefdata}\SpecialCharTok{::}\NormalTok{relief\_humac }\CommentTok{\# this is the peak torque test data}

\NormalTok{condition }\OtherTok{\textless{}{-}}\NormalTok{ reliefdata}\SpecialCharTok{::}\NormalTok{relief\_volume }\SpecialCharTok{\%\textgreater{}\%} \CommentTok{\# this is info about volume condition}
  \FunctionTok{mutate}\NormalTok{(}\AttributeTok{participant =} \FunctionTok{as.character}\NormalTok{(participant))}

\NormalTok{participants }\OtherTok{\textless{}{-}}\NormalTok{ reliefdata}\SpecialCharTok{::}\NormalTok{relief\_participants }\SpecialCharTok{\%\textgreater{}\%} \CommentTok{\# this is info about age group and allocation}
  
  \FunctionTok{mutate}\NormalTok{(}\AttributeTok{age\_group =} \FunctionTok{if\_else}\NormalTok{(age }\SpecialCharTok{\textless{}} \DecValTok{40}\NormalTok{, }\StringTok{"yng"}\NormalTok{, }\StringTok{"old"}\NormalTok{))}

\CommentTok{\# It would probably be a good idea to make 1 dataset overall from relief\_volum and relief\_participants {-} as a note to self}



\DocumentationTok{\#\# Joining the data sets}

\CommentTok{\# Joining condition and participant info}
\NormalTok{humac.dat }\OtherTok{\textless{}{-}}\NormalTok{ humac.dat }\SpecialCharTok{\%\textgreater{}\%}
  \CommentTok{\# Using left\_join to add the info from condition and participants to humac.dat}
  \FunctionTok{left\_join}\NormalTok{(condition) }\SpecialCharTok{\%\textgreater{}\%}
  \CommentTok{\# There is no "arm" data from this test, so we exclude it}
  \FunctionTok{select}\NormalTok{(}\SpecialCharTok{{-}}\NormalTok{arm) }\SpecialCharTok{\%\textgreater{}\%} 
  
  \FunctionTok{left\_join}\NormalTok{(participants) }\SpecialCharTok{\%\textgreater{}\%}
  
  \FunctionTok{mutate}\NormalTok{(}\AttributeTok{age\_group =} \FunctionTok{factor}\NormalTok{(age\_group, }\AttributeTok{levels =} \FunctionTok{c}\NormalTok{(}\StringTok{"yng"}\NormalTok{, }\StringTok{"old"}\NormalTok{)),}
         \CommentTok{\# using factor() to sort the order of the respective variables}
         \AttributeTok{condition =} \FunctionTok{factor}\NormalTok{(condition, }\AttributeTok{levels =} \FunctionTok{c}\NormalTok{(}\StringTok{"low"}\NormalTok{, }\StringTok{"mod"}\NormalTok{)),}
         
         \AttributeTok{time =} \FunctionTok{factor}\NormalTok{(time, }\AttributeTok{levels =} \FunctionTok{c}\NormalTok{(}\StringTok{"pre"}\NormalTok{, }\StringTok{"mid"}\NormalTok{,  }\StringTok{"post"}\NormalTok{)),}
         
         \AttributeTok{speed =} \FunctionTok{factor}\NormalTok{(speed, }\AttributeTok{levels =} \FunctionTok{c}\NormalTok{(}\StringTok{"0"}\NormalTok{, }\StringTok{"60"}\NormalTok{, }\StringTok{"120"}\NormalTok{, }\StringTok{"240"}\NormalTok{)),}
         
         \AttributeTok{allocation =} \FunctionTok{factor}\NormalTok{(allocation, }\AttributeTok{levels =} \FunctionTok{c}\NormalTok{(}\StringTok{"int"}\NormalTok{, }\StringTok{"con"}\NormalTok{)),}
         \CommentTok{\# case\_when makes sure that control participants are not grouped with a volum                condition}
         \AttributeTok{condition =} \FunctionTok{case\_when}\NormalTok{(participant }\SpecialCharTok{\%in\%} \FunctionTok{c}\NormalTok{(}\StringTok{"3071"}\NormalTok{, }\StringTok{"3075"}\NormalTok{, }\StringTok{"3076"}\NormalTok{, }
                                                  \StringTok{"3083"}\NormalTok{, }\StringTok{"3084"}\NormalTok{, }\StringTok{"3085"}\NormalTok{,}
                                                  \StringTok{"3086"}\NormalTok{, }\StringTok{"3090"}\NormalTok{, }\StringTok{"3091"}\NormalTok{,}
                                                  \StringTok{"3094"}\NormalTok{) }\SpecialCharTok{\textasciitilde{}} \StringTok{"none"}\NormalTok{, }\ConstantTok{TRUE} \SpecialCharTok{\textasciitilde{}}\NormalTok{ condition))}
\end{Highlighting}
\end{Shaded}

\begin{verbatim}
Joining with `by = join_by(participant, leg)`
Joining with `by = join_by(participant)`
\end{verbatim}

\begin{Shaded}
\begin{Highlighting}[]
\DocumentationTok{\#\# Further data wrangling}
\CommentTok{\# The data set contains 2x test per time point, so I\textquotesingle{}ve decided to take the highest value observed at each time point for analysis. Another option would be to take the average from each time point. Getting the highest observed measurement from each time point could be done using the summarise() function alone, however, this produced alot of NA\textquotesingle{}s and corresponding errors. Therefore I\textquotesingle{}ve included the below code, which makes a functions that returns proper NA values where all values are NA for a group.}


\NormalTok{safe\_max }\OtherTok{\textless{}{-}} \ControlFlowTok{function}\NormalTok{(x) \{}
  \ControlFlowTok{if}\NormalTok{ (}\FunctionTok{all}\NormalTok{(}\FunctionTok{is.na}\NormalTok{(x))) }\FunctionTok{return}\NormalTok{(}\ConstantTok{NA\_real\_}\NormalTok{)}
  \FunctionTok{max}\NormalTok{(x, }\AttributeTok{na.rm =} \ConstantTok{TRUE}\NormalTok{)}
\NormalTok{\}}

\NormalTok{safe\_min }\OtherTok{\textless{}{-}} \ControlFlowTok{function}\NormalTok{(x) \{}
  \ControlFlowTok{if}\NormalTok{ (}\FunctionTok{all}\NormalTok{(}\FunctionTok{is.na}\NormalTok{(x))) }\FunctionTok{return}\NormalTok{(}\ConstantTok{NA\_real\_}\NormalTok{)}
  \FunctionTok{min}\NormalTok{(x, }\AttributeTok{na.rm =} \ConstantTok{TRUE}\NormalTok{)}
\NormalTok{\}}


\DocumentationTok{\#\# Step{-}by{-}step explanation:}
\CommentTok{\#1. function(x) {-} This creates a function that takes one input called x (which will be your vector of values)}
\CommentTok{\#2. if (all(is.na(x))) {-} This checks the condition:}

\CommentTok{\#is.na(x) checks each element and returns TRUE/FALSE for each}
\CommentTok{\#all() checks if ALL elements are TRUE}
\CommentTok{\#So this asks: "Are ALL values in x missing (NA)?"}

\CommentTok{\#3. return(NA\_real\_) {-} If all values are NA:}

\CommentTok{\#Return NA\_real\_ (which is specifically a numeric NA, as opposed to NA\_character\_ or NA\_integer\_)}
\CommentTok{\#return() immediately exits the function}

\CommentTok{\#4. max(x, na.rm = TRUE) {-} If NOT all values are NA:}

\CommentTok{\#Calculate the maximum, ignoring any NAs that exist}
\CommentTok{\#This line only runs if the if condition is FALSE}

\DocumentationTok{\#\# Get max values for all outcomes}
\CommentTok{\# With the new "safe\_max" variable, we can safely get the highest measurements without producing erroneous NAs.}

\NormalTok{max.dat }\OtherTok{\textless{}{-}}\NormalTok{ humac.dat }\SpecialCharTok{\%\textgreater{}\%}
  \FunctionTok{filter}\NormalTok{(}\SpecialCharTok{!}\NormalTok{participant }\SpecialCharTok{==} \StringTok{"3012"}\NormalTok{) }\SpecialCharTok{\%\textgreater{}\%} \CommentTok{\# Missing data}
  \FunctionTok{group\_by}\NormalTok{(participant, time, leg, speed, condition, age\_group, allocation, sex) }\SpecialCharTok{\%\textgreater{}\%}
  \FunctionTok{summarise}\NormalTok{(}\AttributeTok{pt\_max =} \FunctionTok{safe\_max}\NormalTok{(pt),}
            \AttributeTok{power\_max =} \FunctionTok{safe\_max}\NormalTok{(rep\_power),}
            \AttributeTok{work\_max =} \FunctionTok{safe\_max}\NormalTok{(rep\_work),}
            \AttributeTok{tt\_min =} \FunctionTok{safe\_min}\NormalTok{(pt\_tt),}
            \AttributeTok{angle =} \FunctionTok{mean}\NormalTok{(pt\_angle, }\AttributeTok{na.rn =} \ConstantTok{TRUE}\NormalTok{),}
            \AttributeTok{.groups =} \StringTok{"drop"}\NormalTok{)}


\DocumentationTok{\#\# Reshape the data to long format for faceting}
\CommentTok{\# This simply formats the data to where each observation is stored in its own row, with variables spread}
\CommentTok{\# across columns {-} thus minimizing redundancy. }

\NormalTok{long.dat }\OtherTok{\textless{}{-}}\NormalTok{ max.dat }\SpecialCharTok{\%\textgreater{}\%}
  \FunctionTok{pivot\_longer}\NormalTok{(}\AttributeTok{names\_to =} \StringTok{"outcome"}\NormalTok{,}
               \AttributeTok{values\_to =} \StringTok{"value"}\NormalTok{,}
               \AttributeTok{cols =} \FunctionTok{c}\NormalTok{(pt\_max, power\_max, work\_max, tt\_min, angle))}

\DocumentationTok{\#\# Filter by speed and parameter}
\CommentTok{\# Below separate data sets for the different velocities are created with the filter() function, also filtering to only the peak torque measurement, since thats what Im interested in here. This could also be done in the plot itself, but I prefer it this way to minimize the amount of code that goes into the actual plot. This might be a personal preference, and there\textquotesingle{}s probably better ways to do it. }

\CommentTok{\# Isometric/ 0 d/s}

\NormalTok{ptisom }\OtherTok{\textless{}{-}}\NormalTok{ long.dat }\SpecialCharTok{\%\textgreater{}\%}
  \FunctionTok{filter}\NormalTok{(speed }\SpecialCharTok{==} \StringTok{"0"}\NormalTok{, outcome }\SpecialCharTok{==} \StringTok{"pt\_max"}\NormalTok{)}


\CommentTok{\# 60 d/s}

\NormalTok{pt60 }\OtherTok{\textless{}{-}}\NormalTok{ long.dat }\SpecialCharTok{\%\textgreater{}\%}
  \FunctionTok{filter}\NormalTok{(speed }\SpecialCharTok{==} \StringTok{"60"}\NormalTok{, outcome }\SpecialCharTok{==} \StringTok{"pt\_max"}\NormalTok{)}

  
\CommentTok{\# 120 d/s}

\NormalTok{pt120 }\OtherTok{\textless{}{-}}\NormalTok{ long.dat }\SpecialCharTok{\%\textgreater{}\%}
  \FunctionTok{filter}\NormalTok{(speed }\SpecialCharTok{==} \StringTok{"120"}\NormalTok{, outcome }\SpecialCharTok{==} \StringTok{"pt\_max"}\NormalTok{)}


\CommentTok{\# 240 d/s}

\NormalTok{pt240 }\OtherTok{\textless{}{-}}\NormalTok{ long.dat }\SpecialCharTok{\%\textgreater{}\%}
  \FunctionTok{filter}\NormalTok{(speed }\SpecialCharTok{==} \StringTok{"240"}\NormalTok{, outcome }\SpecialCharTok{==} \StringTok{"pt\_max"}\NormalTok{)}
\end{Highlighting}
\end{Shaded}

\subsubsection{Plotting some figures}\label{plotting-some-figures}

Now that the data is tidied up for our purpose, we can make some
different plots. The backbone of these plots will be the ggplot()
function from ggplot2 (also loaded by tidyverse), and in addition we
will also use cowplot for some more toys. Briefly, ggplot2

\textbf{Plot 1: Individual Changes in Peak Torque at 0 (isometric), 60,
120 and 240 d/S}

\begin{Shaded}
\begin{Highlighting}[]
\FunctionTok{library}\NormalTok{(cowplot)}
\end{Highlighting}
\end{Shaded}

\begin{verbatim}

Attaching package: 'cowplot'
\end{verbatim}

\begin{verbatim}
The following object is masked from 'package:lubridate':

    stamp
\end{verbatim}

\begin{Shaded}
\begin{Highlighting}[]
\DocumentationTok{\#\# Peak torque Isometric}

\NormalTok{ptisom.plot }\OtherTok{\textless{}{-}}\NormalTok{ ptisom }\SpecialCharTok{\%\textgreater{}\%}
  \FunctionTok{ggplot}\NormalTok{(}\FunctionTok{aes}\NormalTok{(}\AttributeTok{x =}\NormalTok{ time, }\AttributeTok{y =}\NormalTok{ value, }\AttributeTok{group =} \FunctionTok{interaction}\NormalTok{(participant, leg))) }\SpecialCharTok{+}
  \FunctionTok{geom\_line}\NormalTok{(}\FunctionTok{aes}\NormalTok{(}\AttributeTok{color =}\NormalTok{ condition), }\AttributeTok{alpha =} \FloatTok{0.3}\NormalTok{) }\SpecialCharTok{+}
  \FunctionTok{geom\_point}\NormalTok{(}\FunctionTok{aes}\NormalTok{(}\AttributeTok{color =}\NormalTok{ condition), }\AttributeTok{alpha =} \FloatTok{0.3}\NormalTok{, }\AttributeTok{size =} \DecValTok{1}\NormalTok{) }\SpecialCharTok{+}
  \FunctionTok{facet\_wrap}\NormalTok{(}\FunctionTok{vars}\NormalTok{(age\_group, allocation), }\AttributeTok{scales =} \StringTok{"fixed"}\NormalTok{) }\SpecialCharTok{+}
  \FunctionTok{scale\_color\_manual}\NormalTok{(}\AttributeTok{values =} \FunctionTok{c}\NormalTok{(}\StringTok{"low"} \OtherTok{=} \StringTok{"\#E69F00"}\NormalTok{, }\StringTok{"mod"} \OtherTok{=} \StringTok{"\#56B4E9"}\NormalTok{, }\StringTok{"none"} \OtherTok{=} \StringTok{"gray50"}\NormalTok{),}
                     \AttributeTok{name =} \StringTok{"Training Volume"}\NormalTok{) }\SpecialCharTok{+}
  \FunctionTok{labs}\NormalTok{(}\AttributeTok{title =} \StringTok{""}\NormalTok{,}
       \AttributeTok{subtitle =} \StringTok{""}\NormalTok{,}
       \AttributeTok{x =} \StringTok{""}\NormalTok{,}
       \AttributeTok{y =} \StringTok{"Peak Torque (Nm)"}\NormalTok{) }\SpecialCharTok{+}
  \FunctionTok{theme\_minimal}\NormalTok{() }\SpecialCharTok{+}
  \FunctionTok{theme}\NormalTok{(}\AttributeTok{legend.position =} \StringTok{"bottom"}\NormalTok{,}
        \AttributeTok{panel.grid.major =} \FunctionTok{element\_blank}\NormalTok{(),}
        \AttributeTok{panel.grid.minor =} \FunctionTok{element\_blank}\NormalTok{(),}
        \AttributeTok{axis.line =} \FunctionTok{element\_line}\NormalTok{(}\AttributeTok{color =} \StringTok{"black"}\NormalTok{),}
        \AttributeTok{axis.line.x =} \FunctionTok{element\_blank}\NormalTok{(),}
        \AttributeTok{axis.text.x =} \FunctionTok{element\_blank}\NormalTok{(),}
        \AttributeTok{strip.text =} \FunctionTok{element\_blank}\NormalTok{())}

\DocumentationTok{\#\# Peak torque 60}

\NormalTok{pt60.plot }\OtherTok{\textless{}{-}}\NormalTok{ pt60 }\SpecialCharTok{\%\textgreater{}\%}
  \FunctionTok{ggplot}\NormalTok{(}\FunctionTok{aes}\NormalTok{(}\AttributeTok{x =}\NormalTok{ time, }\AttributeTok{y =}\NormalTok{ value, }\AttributeTok{group =} \FunctionTok{interaction}\NormalTok{(participant, leg))) }\SpecialCharTok{+}
  \FunctionTok{geom\_line}\NormalTok{(}\FunctionTok{aes}\NormalTok{(}\AttributeTok{color =}\NormalTok{ condition), }\AttributeTok{alpha =} \FloatTok{0.3}\NormalTok{) }\SpecialCharTok{+}
  \FunctionTok{geom\_point}\NormalTok{(}\FunctionTok{aes}\NormalTok{(}\AttributeTok{color =}\NormalTok{ condition), }\AttributeTok{alpha =} \FloatTok{0.3}\NormalTok{, }\AttributeTok{size =} \DecValTok{1}\NormalTok{) }\SpecialCharTok{+}
  \FunctionTok{facet\_wrap}\NormalTok{(}\FunctionTok{vars}\NormalTok{(age\_group, allocation), }\AttributeTok{scales =} \StringTok{"fixed"}\NormalTok{) }\SpecialCharTok{+}
  \FunctionTok{scale\_color\_manual}\NormalTok{(}\AttributeTok{values =} \FunctionTok{c}\NormalTok{(}\StringTok{"low"} \OtherTok{=} \StringTok{"\#E69F00"}\NormalTok{, }\StringTok{"mod"} \OtherTok{=} \StringTok{"\#56B4E9"}\NormalTok{, }\StringTok{"none"} \OtherTok{=} \StringTok{"gray50"}\NormalTok{),}
                     \AttributeTok{name =} \StringTok{"Training Volume"}\NormalTok{) }\SpecialCharTok{+}
  \FunctionTok{labs}\NormalTok{(}\AttributeTok{title =} \StringTok{""}\NormalTok{,}
       \AttributeTok{subtitle =} \StringTok{""}\NormalTok{,}
       \AttributeTok{x =} \StringTok{""}\NormalTok{,}
       \AttributeTok{y =} \StringTok{""}\NormalTok{) }\SpecialCharTok{+}
  \FunctionTok{theme\_minimal}\NormalTok{() }\SpecialCharTok{+}
  \FunctionTok{theme}\NormalTok{(}\AttributeTok{legend.position =} \StringTok{"bottom"}\NormalTok{) }\SpecialCharTok{+}
  \FunctionTok{theme}\NormalTok{(}\AttributeTok{panel.grid.major =} \FunctionTok{element\_blank}\NormalTok{(),}
        \AttributeTok{panel.grid.minor =} \FunctionTok{element\_blank}\NormalTok{(),}
        \AttributeTok{axis.line =} \FunctionTok{element\_line}\NormalTok{(}\AttributeTok{color =} \StringTok{"black"}\NormalTok{),}
        \AttributeTok{axis.line.x =} \FunctionTok{element\_blank}\NormalTok{(),}
        \AttributeTok{axis.text.x =} \FunctionTok{element\_blank}\NormalTok{())}

\DocumentationTok{\#\# Peak torque 120}

\NormalTok{pt120.plot }\OtherTok{\textless{}{-}}\NormalTok{ pt120 }\SpecialCharTok{\%\textgreater{}\%}
  \FunctionTok{ggplot}\NormalTok{(}\FunctionTok{aes}\NormalTok{(}\AttributeTok{x =}\NormalTok{ time, }\AttributeTok{y =}\NormalTok{ value, }\AttributeTok{group =} \FunctionTok{interaction}\NormalTok{(participant, leg))) }\SpecialCharTok{+}
  \FunctionTok{geom\_line}\NormalTok{(}\FunctionTok{aes}\NormalTok{(}\AttributeTok{color =}\NormalTok{ condition), }\AttributeTok{alpha =} \FloatTok{0.3}\NormalTok{) }\SpecialCharTok{+}
  \FunctionTok{geom\_point}\NormalTok{(}\FunctionTok{aes}\NormalTok{(}\AttributeTok{color =}\NormalTok{ condition), }\AttributeTok{alpha =} \FloatTok{0.3}\NormalTok{, }\AttributeTok{size =} \DecValTok{1}\NormalTok{) }\SpecialCharTok{+}
  \FunctionTok{facet\_wrap}\NormalTok{(}\FunctionTok{vars}\NormalTok{(age\_group, allocation), }\AttributeTok{scales =} \StringTok{"fixed"}\NormalTok{) }\SpecialCharTok{+}
  \FunctionTok{scale\_color\_manual}\NormalTok{(}\AttributeTok{values =} \FunctionTok{c}\NormalTok{(}\StringTok{"low"} \OtherTok{=} \StringTok{"\#E69F00"}\NormalTok{, }\StringTok{"mod"} \OtherTok{=} \StringTok{"\#56B4E9"}\NormalTok{, }\StringTok{"none"} \OtherTok{=} \StringTok{"gray50"}\NormalTok{),}
                     \AttributeTok{name =} \StringTok{"Training Volume"}\NormalTok{) }\SpecialCharTok{+}
  \FunctionTok{labs}\NormalTok{(}\AttributeTok{title =} \StringTok{""}\NormalTok{,}
       \AttributeTok{subtitle =} \StringTok{""}\NormalTok{,}
       \AttributeTok{x =} \StringTok{"Time"}\NormalTok{,}
       \AttributeTok{y =} \StringTok{"Peak Torque (Nm)"}\NormalTok{) }\SpecialCharTok{+}
  \FunctionTok{theme\_minimal}\NormalTok{() }\SpecialCharTok{+}
  \FunctionTok{theme}\NormalTok{(}\AttributeTok{legend.position =} \StringTok{"bottom"}\NormalTok{) }\SpecialCharTok{+}
  \FunctionTok{theme}\NormalTok{(}\AttributeTok{panel.grid.major =} \FunctionTok{element\_blank}\NormalTok{(),}
        \AttributeTok{panel.grid.minor =} \FunctionTok{element\_blank}\NormalTok{(),}
        \AttributeTok{axis.line =} \FunctionTok{element\_line}\NormalTok{(}\AttributeTok{color =} \StringTok{"black"}\NormalTok{))}

\DocumentationTok{\#\# Peak torque 240}

\NormalTok{pt240.plot }\OtherTok{\textless{}{-}}\NormalTok{ pt240 }\SpecialCharTok{\%\textgreater{}\%}
  \FunctionTok{ggplot}\NormalTok{(}\FunctionTok{aes}\NormalTok{(}\AttributeTok{x =}\NormalTok{ time, }\AttributeTok{y =}\NormalTok{ value, }\AttributeTok{group =} \FunctionTok{interaction}\NormalTok{(participant, leg))) }\SpecialCharTok{+}
  \FunctionTok{geom\_line}\NormalTok{(}\FunctionTok{aes}\NormalTok{(}\AttributeTok{color =}\NormalTok{ condition), }\AttributeTok{alpha =} \FloatTok{0.3}\NormalTok{) }\SpecialCharTok{+}
  \FunctionTok{geom\_point}\NormalTok{(}\FunctionTok{aes}\NormalTok{(}\AttributeTok{color =}\NormalTok{ condition), }\AttributeTok{alpha =} \FloatTok{0.3}\NormalTok{, }\AttributeTok{size =} \DecValTok{1}\NormalTok{) }\SpecialCharTok{+}
  \FunctionTok{facet\_wrap}\NormalTok{(}\FunctionTok{vars}\NormalTok{(age\_group, allocation), }\AttributeTok{scales =} \StringTok{"fixed"}\NormalTok{) }\SpecialCharTok{+}
  \FunctionTok{scale\_color\_manual}\NormalTok{(}\AttributeTok{values =} \FunctionTok{c}\NormalTok{(}\StringTok{"low"} \OtherTok{=} \StringTok{"\#E69F00"}\NormalTok{, }\StringTok{"mod"} \OtherTok{=} \StringTok{"\#56B4E9"}\NormalTok{, }\StringTok{"none"} \OtherTok{=} \StringTok{"gray50"}\NormalTok{),}
                     \AttributeTok{name =} \StringTok{"Training Volume"}\NormalTok{) }\SpecialCharTok{+}
  \FunctionTok{labs}\NormalTok{(}\AttributeTok{title =} \StringTok{""}\NormalTok{,}
       \AttributeTok{subtitle =} \StringTok{""}\NormalTok{, }\CommentTok{\#Each line represents one leg from one participant}
       \AttributeTok{x =} \StringTok{"Time"}\NormalTok{,}
       \AttributeTok{y =} \StringTok{""}\NormalTok{) }\SpecialCharTok{+}
  \FunctionTok{theme\_minimal}\NormalTok{() }\SpecialCharTok{+}
  \FunctionTok{theme}\NormalTok{(}\AttributeTok{legend.position =} \StringTok{"bottom"}\NormalTok{,}
        \AttributeTok{legend.box.margin =} \FunctionTok{margin}\NormalTok{(}\DecValTok{0}\NormalTok{, }\DecValTok{0}\NormalTok{, }\DecValTok{0}\NormalTok{, }\DecValTok{0}\NormalTok{)) }\SpecialCharTok{+}
  \FunctionTok{theme}\NormalTok{(}\AttributeTok{panel.grid.major =} \FunctionTok{element\_blank}\NormalTok{(),}
        \AttributeTok{panel.grid.minor =} \FunctionTok{element\_blank}\NormalTok{(),}
        \AttributeTok{axis.line =} \FunctionTok{element\_line}\NormalTok{(}\AttributeTok{color =} \StringTok{"black"}\NormalTok{))}




\CommentTok{\# Combinding the plots}

\NormalTok{ptfig }\OtherTok{\textless{}{-}} \FunctionTok{plot\_grid}\NormalTok{(}
\NormalTok{  ptisom.plot }\SpecialCharTok{+} \FunctionTok{theme}\NormalTok{(}\AttributeTok{legend.position =} \StringTok{"none"}\NormalTok{),}
\NormalTok{  pt60.plot }\SpecialCharTok{+} \FunctionTok{theme}\NormalTok{(}\AttributeTok{legend.position =} \StringTok{"none"}\NormalTok{),}
\NormalTok{  pt120.plot,}
\NormalTok{  pt240.plot,}
  \AttributeTok{ncol =} \DecValTok{2}\NormalTok{,}
  \AttributeTok{nrow =} \DecValTok{2}\NormalTok{,}
  \AttributeTok{labels =} \FunctionTok{c}\NormalTok{(}\StringTok{"A) 0 d/s"}\NormalTok{, }\StringTok{"B) 60 d/s"}\NormalTok{, }\StringTok{"C) 120 d/s"}\NormalTok{, }\StringTok{"D) 240 d/s"}\NormalTok{),}
  \AttributeTok{rel\_heights =} \FunctionTok{c}\NormalTok{(}\FloatTok{0.8}\NormalTok{, }\DecValTok{1}\NormalTok{) }
\NormalTok{)}


\NormalTok{ptfig}
\end{Highlighting}
\end{Shaded}

\pandocbounded{\includegraphics[keepaspectratio]{manuscript_files/figure-pdf/Plot1-1.pdf}}

\begin{Shaded}
\begin{Highlighting}[]
\DocumentationTok{\#\# Isometric }
\CommentTok{\#}
\CommentTok{\#ptisommean \textless{}{-} ptisom \%\textgreater{}\%}
\CommentTok{\#  \#summarising means}
\CommentTok{\#  group\_by(age\_group, allocation, condition, time) \%\textgreater{}\%}
\CommentTok{\#  summarise(mean = mean(value, na.rm = TRUE),}
\CommentTok{\#            sd = sd(value, na.rm = TRUE),}
\CommentTok{\#            .groups = "drop" ) \%\textgreater{}\%}
\CommentTok{\#  }
\CommentTok{\#  \#creating the plot}
\CommentTok{\#  ggplot(aes(x = time, y = mean, color = condition, group = condition)) +}
\CommentTok{\#  }
\CommentTok{\#  geom\_line(linewidth = 1, position = position\_dodge(.2)) +}
\CommentTok{\#  geom\_point(size = 3, position = position\_dodge(.2)) +}
\CommentTok{\#  geom\_errorbar(aes(ymin = mean {-} sd, ymax = mean + sd),}
\CommentTok{\#                width = 0.1, linewidth = 0.8, position = position\_dodge(.2)) +}
\CommentTok{\#  }
\CommentTok{\#  facet\_wrap(vars(age\_group, allocation), scales = "fixed") +}
\CommentTok{\#  }
\CommentTok{\#  scale\_color\_manual(values = c("low" = "\#E69F00", "mod" = "\#56B4E9", "none" = "gray50"),}
\CommentTok{\#                     name = "Training Volume") +}
\CommentTok{\#  }
\CommentTok{\#  labs(title = "",}
\CommentTok{\#       subtitle = "",}
\CommentTok{\#       x = "",}
\CommentTok{\#       y = "Peak Torque (Nm)") +}
\CommentTok{\#  }
\CommentTok{\#  theme\_minimal() +}
\CommentTok{\#  theme(legend.position = "bottom") +}
\CommentTok{\#  theme(panel.grid.major = element\_blank(),}
\CommentTok{\#        panel.grid.minor = element\_blank(),}
\CommentTok{\#        axis.line = element\_line(color = "black"),}
\CommentTok{\#        axis.text.x = element\_blank(),}
\CommentTok{\#        axis.line.x = element\_blank())}
\CommentTok{\#}
\CommentTok{\#}
\DocumentationTok{\#\# 60 d/s}
\CommentTok{\#pt60mean \textless{}{-} pt60 \%\textgreater{}\%}
\CommentTok{\#  \#summarising means}
\CommentTok{\#  group\_by(age\_group, allocation, condition, time) \%\textgreater{}\%}
\CommentTok{\#  summarise(mean = mean(value, na.rm = TRUE),}
\CommentTok{\#            sd = sd(value, na.rm = TRUE),}
\CommentTok{\#            .groups = "drop" ) \%\textgreater{}\%}
\CommentTok{\#  }
\CommentTok{\#  \#creating the plot}
\CommentTok{\#  ggplot(aes(x = time, y = mean, color = condition, group = condition)) +}
\CommentTok{\#  }
\CommentTok{\#  geom\_line(linewidth = 1, position = position\_dodge(.2)) +}
\CommentTok{\#  geom\_point(size = 3, position = position\_dodge(.2)) +}
\CommentTok{\#  geom\_errorbar(aes(ymin = mean {-} sd, ymax = mean + sd),}
\CommentTok{\#                width = 0.1, linewidth = 0.8, position = position\_dodge(.2)) +}
\CommentTok{\#  }
\CommentTok{\#  facet\_wrap(vars(age\_group, allocation), scales = "fixed") +}
\CommentTok{\#  }
\CommentTok{\#  scale\_color\_manual(values = c("low" = "\#E69F00", "mod" = "\#56B4E9", "none" = "gray50"),}
\CommentTok{\#                     name = "Training Volume") +}
\CommentTok{\#  }
\CommentTok{\#  labs(title = "",}
\CommentTok{\#       subtitle = "",}
\CommentTok{\#       x = "",}
\CommentTok{\#       y = "") +}
\CommentTok{\#  }
\CommentTok{\#  theme\_minimal() +}
\CommentTok{\#  theme(legend.position = "bottom") +}
\CommentTok{\#  theme(panel.grid.major = element\_blank(),}
\CommentTok{\#        panel.grid.minor = element\_blank(),}
\CommentTok{\#        axis.line = element\_line(color = "black"),}
\CommentTok{\#        axis.text.x = element\_blank(),}
\CommentTok{\#        axis.line.x = element\_blank())}
\CommentTok{\#}
\CommentTok{\#}
\DocumentationTok{\#\# 120 d/s}
\CommentTok{\#pt120mean \textless{}{-} pt120 \%\textgreater{}\%}
\CommentTok{\#  \#summarising means}
\CommentTok{\#  group\_by(age\_group, allocation, condition, time) \%\textgreater{}\%}
\CommentTok{\#  summarise(mean = mean(value, na.rm = TRUE),}
\CommentTok{\#            sd = sd(value, na.rm = TRUE),}
\CommentTok{\#            .groups = "drop" ) \%\textgreater{}\%}
\CommentTok{\#  }
\CommentTok{\#  \#creating the plot}
\CommentTok{\#  ggplot(aes(x = time, y = mean, color = condition, group = condition)) +}
\CommentTok{\#  }
\CommentTok{\#  geom\_line(linewidth = 1, position = position\_dodge(.2)) +}
\CommentTok{\#  geom\_point(size = 3, position = position\_dodge(.2)) +}
\CommentTok{\#  geom\_errorbar(aes(ymin = mean {-} sd, ymax = mean + sd),}
\CommentTok{\#                width = 0.1, linewidth = 0.8, position = position\_dodge(.2)) +}
\CommentTok{\#  }
\CommentTok{\#  facet\_wrap(vars(age\_group, allocation), scales = "fixed") +}
\CommentTok{\#  }
\CommentTok{\#  scale\_color\_manual(values = c("low" = "\#E69F00", "mod" = "\#56B4E9", "none" = "gray50"),}
\CommentTok{\#                     name = "Training Volume") +}
\CommentTok{\#  }
\CommentTok{\#  labs(title = "",}
\CommentTok{\#       subtitle = "",}
\CommentTok{\#       x = "Time",}
\CommentTok{\#       y = "Peak Torque (Nm)") +}
\CommentTok{\#  }
\CommentTok{\#  theme\_minimal() +}
\CommentTok{\#  theme(legend.position = "bottom") +}
\CommentTok{\#  theme(panel.grid.major = element\_blank(),}
\CommentTok{\#        panel.grid.minor = element\_blank(),}
\CommentTok{\#        axis.line = element\_line(color = "black"))}
\CommentTok{\#}
\CommentTok{\#}
\DocumentationTok{\#\# 120 d/s}
\CommentTok{\#pt240mean \textless{}{-} pt240 \%\textgreater{}\%}
\CommentTok{\#  \#summarising means}
\CommentTok{\#  group\_by(age\_group, allocation, condition, time) \%\textgreater{}\%}
\CommentTok{\#  summarise(mean = mean(value, na.rm = TRUE),}
\CommentTok{\#            sd = sd(value, na.rm = TRUE),}
\CommentTok{\#            .groups = "drop" ) \%\textgreater{}\%}
\CommentTok{\#  }
\CommentTok{\#  \#creating the plot}
\CommentTok{\#  ggplot(aes(x = time, y = mean, color = condition, group = condition)) +}
\CommentTok{\#  }
\CommentTok{\#  geom\_line(linewidth = 1, position = position\_dodge(.2)) +}
\CommentTok{\#  geom\_point(size = 3, position = position\_dodge(.2)) +}
\CommentTok{\#  geom\_errorbar(aes(ymin = mean {-} sd, ymax = mean + sd),}
\CommentTok{\#                width = 0.1, linewidth = 0.8, position = position\_dodge(.2)) +}
\CommentTok{\#  }
\CommentTok{\#  facet\_wrap(vars(age\_group, allocation), scales = "fixed") +}
\CommentTok{\#  }
\CommentTok{\#  scale\_color\_manual(values = c("low" = "\#E69F00", "mod" = "\#56B4E9", "none" = "gray50"),}
\CommentTok{\#                     name = "Training Volume") +}
\CommentTok{\#  }
\CommentTok{\#  labs(title = "",}
\CommentTok{\#       subtitle = "",}
\CommentTok{\#       x = "Time",}
\CommentTok{\#       y = "") +}
\CommentTok{\#  }
\CommentTok{\#  theme\_minimal() +}
\CommentTok{\#  theme(legend.position = "bottom") +}
\CommentTok{\#  theme(panel.grid.major = element\_blank(),}
\CommentTok{\#        panel.grid.minor = element\_blank(),}
\CommentTok{\#        axis.line = element\_line(color = "black"))}
\CommentTok{\#}
\CommentTok{\#}
\DocumentationTok{\#\# Combinding the plots}
\CommentTok{\#}
\CommentTok{\#ptmeanfig \textless{}{-} plot\_grid(}
\CommentTok{\#  ptisommean + theme(legend.position = "none"),}
\CommentTok{\#  pt60mean + theme(legend.position = "none"),}
\CommentTok{\#  pt120mean,}
\CommentTok{\#  pt240mean,}
\CommentTok{\#  ncol = 2,}
\CommentTok{\#  nrow = 2,}
\CommentTok{\#  labels = c("A) Isometric", "B) 60 d/s", "C) 120 d/s", "D) 240 d/s"),}
\CommentTok{\#  rel\_heights = c(0.8, 1) }
\CommentTok{\#)}
\CommentTok{\#}
\CommentTok{\#}
\CommentTok{\#ptmeanfig}
\CommentTok{\#}
\CommentTok{\#}
\end{Highlighting}
\end{Shaded}

\textbf{Plot 3: Mean Change Peak Torque per Age Group - with raw data}

\begin{Shaded}
\begin{Highlighting}[]
\CommentTok{\# Set seed for jitter}

\FunctionTok{set.seed}\NormalTok{(}\DecValTok{1}\NormalTok{)}

\CommentTok{\# Isometric }

\NormalTok{ptiisommean }\OtherTok{\textless{}{-}}\NormalTok{ ptisom }\SpecialCharTok{\%\textgreater{}\%}
  \CommentTok{\#summarising means}
  \FunctionTok{group\_by}\NormalTok{(age\_group, allocation, condition, time) }\SpecialCharTok{\%\textgreater{}\%}
  \FunctionTok{summarise}\NormalTok{(}\AttributeTok{mean =} \FunctionTok{mean}\NormalTok{(value, }\AttributeTok{na.rm =} \ConstantTok{TRUE}\NormalTok{),}
            \AttributeTok{sd =} \FunctionTok{sd}\NormalTok{(value, }\AttributeTok{na.rm =} \ConstantTok{TRUE}\NormalTok{),}
            \AttributeTok{.groups =} \StringTok{"drop"}\NormalTok{ ) }\SpecialCharTok{\%\textgreater{}\%}
  
  \CommentTok{\#creating the plot}
  \FunctionTok{ggplot}\NormalTok{(}\FunctionTok{aes}\NormalTok{(}\AttributeTok{x =}\NormalTok{ time, }\AttributeTok{y =}\NormalTok{ mean, }\AttributeTok{color =}\NormalTok{ condition, }\AttributeTok{group =}\NormalTok{ condition)) }\SpecialCharTok{+}
  
  \FunctionTok{geom\_line}\NormalTok{(}\AttributeTok{linewidth =} \DecValTok{1}\NormalTok{, }\AttributeTok{position =} \FunctionTok{position\_dodge}\NormalTok{(.}\DecValTok{5}\NormalTok{)) }\SpecialCharTok{+}
  \FunctionTok{geom\_point}\NormalTok{(}\AttributeTok{shape =} \DecValTok{21}\NormalTok{, }\FunctionTok{aes}\NormalTok{(}\AttributeTok{fill =}\NormalTok{ condition), }\AttributeTok{color =} \StringTok{"black"}\NormalTok{,}
             \AttributeTok{size =} \DecValTok{3}\NormalTok{, }\AttributeTok{stroke =} \DecValTok{1}\NormalTok{, }\AttributeTok{position =} \FunctionTok{position\_dodge}\NormalTok{(.}\DecValTok{5}\NormalTok{)) }\SpecialCharTok{+}
\CommentTok{\#  geom\_errorbar(aes(ymin = mean {-} sd, ymax = mean + sd),}
\CommentTok{\#                width = 0.1, linewidth = 0.8, position = position\_dodge(.2)) +}

  \CommentTok{\# adding in the raw data}
  \FunctionTok{geom\_point}\NormalTok{(}\AttributeTok{data =}\NormalTok{ ptisom,}
             \FunctionTok{aes}\NormalTok{(}\AttributeTok{x =}\NormalTok{ time, }\AttributeTok{y =}\NormalTok{ value,}
                 \AttributeTok{color =}\NormalTok{ condition, }\AttributeTok{group =}\NormalTok{ condition),}
                 \AttributeTok{size =} \FloatTok{1.5}\NormalTok{,}
                 \AttributeTok{alpha =} \FloatTok{0.3}\NormalTok{,}
             \AttributeTok{position =} \FunctionTok{position\_jitter}\NormalTok{(}\AttributeTok{width =} \FloatTok{0.2}\NormalTok{)) }\SpecialCharTok{+}
  
  \FunctionTok{facet\_wrap}\NormalTok{(}\FunctionTok{vars}\NormalTok{(age\_group, allocation), }\AttributeTok{scales =} \StringTok{"fixed"}\NormalTok{) }\SpecialCharTok{+}
  
  \FunctionTok{scale\_color\_manual}\NormalTok{(}\AttributeTok{values =} \FunctionTok{c}\NormalTok{(}\StringTok{"low"} \OtherTok{=} \StringTok{"\#E69F00"}\NormalTok{, }\StringTok{"mod"} \OtherTok{=} \StringTok{"\#56B4E9"}\NormalTok{, }\StringTok{"none"} \OtherTok{=} \StringTok{"gray50"}\NormalTok{),}
                     \AttributeTok{name =} \StringTok{"Training Volume"}\NormalTok{) }\SpecialCharTok{+}  \CommentTok{\# \textless{}{-}{-} hide the color legend}
  
  \FunctionTok{scale\_fill\_manual}\NormalTok{(}\AttributeTok{values =} \FunctionTok{c}\NormalTok{(}\StringTok{"low"} \OtherTok{=} \StringTok{"\#E69F00"}\NormalTok{, }\StringTok{"mod"} \OtherTok{=} \StringTok{"\#56B4E9"}\NormalTok{, }\StringTok{"none"} \OtherTok{=} \StringTok{"gray50"}\NormalTok{),}
                    \AttributeTok{guide =} \StringTok{"none"}\NormalTok{) }\SpecialCharTok{+}  \CommentTok{\# \textless{}{-}{-} add fill scale}
  
  \FunctionTok{labs}\NormalTok{(}\AttributeTok{title =} \StringTok{""}\NormalTok{,}
       \AttributeTok{subtitle =} \StringTok{""}\NormalTok{,}
       \AttributeTok{x =} \StringTok{""}\NormalTok{,}
       \AttributeTok{y =} \StringTok{"Peak Torque (Nm)"}\NormalTok{) }\SpecialCharTok{+}
  
  \FunctionTok{theme\_minimal}\NormalTok{() }\SpecialCharTok{+}
  \FunctionTok{theme}\NormalTok{(}\AttributeTok{legend.position =} \StringTok{"bottom"}\NormalTok{) }\SpecialCharTok{+}
  \FunctionTok{theme}\NormalTok{(}\AttributeTok{panel.grid.major =} \FunctionTok{element\_blank}\NormalTok{(),}
        \AttributeTok{panel.grid.minor =} \FunctionTok{element\_blank}\NormalTok{(),}
        \AttributeTok{axis.line =} \FunctionTok{element\_line}\NormalTok{(}\AttributeTok{color =} \StringTok{"black"}\NormalTok{),}
        \AttributeTok{axis.text.x =} \FunctionTok{element\_blank}\NormalTok{(),}
        \AttributeTok{axis.line.x =} \FunctionTok{element\_blank}\NormalTok{())}


\CommentTok{\# 60 d/s}
\NormalTok{pti60mean }\OtherTok{\textless{}{-}}\NormalTok{ pt60 }\SpecialCharTok{\%\textgreater{}\%}
  \CommentTok{\#summarising means}
  \FunctionTok{group\_by}\NormalTok{(age\_group, allocation, condition, time) }\SpecialCharTok{\%\textgreater{}\%}
  \FunctionTok{summarise}\NormalTok{(}\AttributeTok{mean =} \FunctionTok{mean}\NormalTok{(value, }\AttributeTok{na.rm =} \ConstantTok{TRUE}\NormalTok{),}
            \AttributeTok{sd =} \FunctionTok{sd}\NormalTok{(value, }\AttributeTok{na.rm =} \ConstantTok{TRUE}\NormalTok{),}
            \AttributeTok{.groups =} \StringTok{"drop"}\NormalTok{ ) }\SpecialCharTok{\%\textgreater{}\%}
  
  \CommentTok{\#creating the plot}
  \FunctionTok{ggplot}\NormalTok{(}\FunctionTok{aes}\NormalTok{(}\AttributeTok{x =}\NormalTok{ time, }\AttributeTok{y =}\NormalTok{ mean, }\AttributeTok{color =}\NormalTok{ condition, }\AttributeTok{group =}\NormalTok{ condition)) }\SpecialCharTok{+}
  
  \FunctionTok{geom\_line}\NormalTok{(}\AttributeTok{linewidth =} \DecValTok{1}\NormalTok{, }\AttributeTok{position =} \FunctionTok{position\_dodge}\NormalTok{(.}\DecValTok{5}\NormalTok{)) }\SpecialCharTok{+}
  \FunctionTok{geom\_point}\NormalTok{(}\AttributeTok{shape =} \DecValTok{21}\NormalTok{, }\FunctionTok{aes}\NormalTok{(}\AttributeTok{fill =}\NormalTok{ condition), }\AttributeTok{color =} \StringTok{"black"}\NormalTok{,}
             \AttributeTok{size =} \DecValTok{3}\NormalTok{, }\AttributeTok{stroke =} \DecValTok{1}\NormalTok{, }\AttributeTok{position =} \FunctionTok{position\_dodge}\NormalTok{(.}\DecValTok{5}\NormalTok{)) }\SpecialCharTok{+}
 \CommentTok{\# geom\_errorbar(aes(ymin = mean {-} sd, ymax = mean + sd),}
\CommentTok{\#                width = 0.1, linewidth = 0.8, position = position\_dodge(.2)) +}
  
  \CommentTok{\# adding in the raw data}
  \FunctionTok{geom\_point}\NormalTok{(}\AttributeTok{data =}\NormalTok{ pt60,}
             \FunctionTok{aes}\NormalTok{(}\AttributeTok{x =}\NormalTok{ time, }\AttributeTok{y =}\NormalTok{ value,}
                 \AttributeTok{color =}\NormalTok{ condition, }\AttributeTok{group =}\NormalTok{ condition),}
                 \AttributeTok{size =} \FloatTok{1.5}\NormalTok{,}
                 \AttributeTok{alpha =} \FloatTok{0.3}\NormalTok{,}
             \AttributeTok{position =} \FunctionTok{position\_jitter}\NormalTok{(}\AttributeTok{width =} \FloatTok{0.2}\NormalTok{)) }\SpecialCharTok{+}
  
  \FunctionTok{facet\_wrap}\NormalTok{(}\FunctionTok{vars}\NormalTok{(age\_group, allocation), }\AttributeTok{scales =} \StringTok{"fixed"}\NormalTok{) }\SpecialCharTok{+}
  
  \FunctionTok{scale\_color\_manual}\NormalTok{(}\AttributeTok{values =} \FunctionTok{c}\NormalTok{(}\StringTok{"low"} \OtherTok{=} \StringTok{"\#E69F00"}\NormalTok{, }\StringTok{"mod"} \OtherTok{=} \StringTok{"\#56B4E9"}\NormalTok{, }\StringTok{"none"} \OtherTok{=} \StringTok{"gray50"}\NormalTok{),}
                     \AttributeTok{name =} \StringTok{"Training Volume"}\NormalTok{) }\SpecialCharTok{+}  \CommentTok{\# \textless{}{-}{-} hide the color legend}
  
  \FunctionTok{scale\_fill\_manual}\NormalTok{(}\AttributeTok{values =} \FunctionTok{c}\NormalTok{(}\StringTok{"low"} \OtherTok{=} \StringTok{"\#E69F00"}\NormalTok{, }\StringTok{"mod"} \OtherTok{=} \StringTok{"\#56B4E9"}\NormalTok{, }\StringTok{"none"} \OtherTok{=} \StringTok{"gray50"}\NormalTok{),}
                    \AttributeTok{guide =} \StringTok{"none"}\NormalTok{) }\SpecialCharTok{+}  \CommentTok{\# \textless{}{-}{-} add fill scale}
  
  \FunctionTok{labs}\NormalTok{(}\AttributeTok{title =} \StringTok{""}\NormalTok{,}
       \AttributeTok{subtitle =} \StringTok{""}\NormalTok{,}
       \AttributeTok{x =} \StringTok{""}\NormalTok{,}
       \AttributeTok{y =} \StringTok{""}\NormalTok{) }\SpecialCharTok{+}
  
  \FunctionTok{theme\_minimal}\NormalTok{() }\SpecialCharTok{+}
  \FunctionTok{theme}\NormalTok{(}\AttributeTok{legend.position =} \StringTok{"bottom"}\NormalTok{) }\SpecialCharTok{+}
  \FunctionTok{theme}\NormalTok{(}\AttributeTok{panel.grid.major =} \FunctionTok{element\_blank}\NormalTok{(),}
        \AttributeTok{panel.grid.minor =} \FunctionTok{element\_blank}\NormalTok{(),}
        \AttributeTok{axis.line =} \FunctionTok{element\_line}\NormalTok{(}\AttributeTok{color =} \StringTok{"black"}\NormalTok{),}
        \AttributeTok{axis.text.x =} \FunctionTok{element\_blank}\NormalTok{(),}
        \AttributeTok{axis.line.x =} \FunctionTok{element\_blank}\NormalTok{())}


\CommentTok{\# 120 d/s}
\NormalTok{pti120mean }\OtherTok{\textless{}{-}}\NormalTok{ pt120 }\SpecialCharTok{\%\textgreater{}\%}
  \CommentTok{\#summarising means}
  \FunctionTok{group\_by}\NormalTok{(age\_group, allocation, condition, time) }\SpecialCharTok{\%\textgreater{}\%}
  \FunctionTok{summarise}\NormalTok{(}\AttributeTok{mean =} \FunctionTok{mean}\NormalTok{(value, }\AttributeTok{na.rm =} \ConstantTok{TRUE}\NormalTok{),}
            \AttributeTok{sd =} \FunctionTok{sd}\NormalTok{(value, }\AttributeTok{na.rm =} \ConstantTok{TRUE}\NormalTok{),}
            \AttributeTok{.groups =} \StringTok{"drop"}\NormalTok{ ) }\SpecialCharTok{\%\textgreater{}\%}
  
  \CommentTok{\#creating the plot}
  \FunctionTok{ggplot}\NormalTok{(}\FunctionTok{aes}\NormalTok{(}\AttributeTok{x =}\NormalTok{ time, }\AttributeTok{y =}\NormalTok{ mean, }\AttributeTok{color =}\NormalTok{ condition, }\AttributeTok{group =}\NormalTok{ condition)) }\SpecialCharTok{+}
  
  \FunctionTok{geom\_line}\NormalTok{(}\AttributeTok{linewidth =} \DecValTok{1}\NormalTok{, }\AttributeTok{position =} \FunctionTok{position\_dodge}\NormalTok{(.}\DecValTok{5}\NormalTok{)) }\SpecialCharTok{+}
  \FunctionTok{geom\_point}\NormalTok{(}\AttributeTok{shape =} \DecValTok{21}\NormalTok{, }\FunctionTok{aes}\NormalTok{(}\AttributeTok{fill =}\NormalTok{ condition), }\AttributeTok{color =} \StringTok{"black"}\NormalTok{,}
             \AttributeTok{size =} \DecValTok{3}\NormalTok{, }\AttributeTok{stroke =} \DecValTok{1}\NormalTok{, }\AttributeTok{position =} \FunctionTok{position\_dodge}\NormalTok{(.}\DecValTok{5}\NormalTok{)) }\SpecialCharTok{+}
 \CommentTok{\# geom\_errorbar(aes(ymin = mean {-} sd, ymax = mean + sd),}
  \CommentTok{\#              width = 0.1, linewidth = 0.8, position = position\_dodge(.2)) +}
  
  \CommentTok{\# adding in the raw data}
  \FunctionTok{geom\_point}\NormalTok{(}\AttributeTok{data =}\NormalTok{ pt120,}
             \FunctionTok{aes}\NormalTok{(}\AttributeTok{x =}\NormalTok{ time, }\AttributeTok{y =}\NormalTok{ value,}
                 \AttributeTok{color =}\NormalTok{ condition, }\AttributeTok{group =}\NormalTok{ condition),}
                 \AttributeTok{size =} \FloatTok{1.5}\NormalTok{,}
                 \AttributeTok{alpha =} \FloatTok{0.3}\NormalTok{,}
             \AttributeTok{position =} \FunctionTok{position\_jitter}\NormalTok{(}\AttributeTok{width =} \FloatTok{0.2}\NormalTok{)) }\SpecialCharTok{+}
  
  \FunctionTok{facet\_wrap}\NormalTok{(}\FunctionTok{vars}\NormalTok{(age\_group, allocation), }\AttributeTok{scales =} \StringTok{"fixed"}\NormalTok{) }\SpecialCharTok{+}
  
  \FunctionTok{scale\_color\_manual}\NormalTok{(}\AttributeTok{values =} \FunctionTok{c}\NormalTok{(}\StringTok{"low"} \OtherTok{=} \StringTok{"\#E69F00"}\NormalTok{, }\StringTok{"mod"} \OtherTok{=} \StringTok{"\#56B4E9"}\NormalTok{, }\StringTok{"none"} \OtherTok{=} \StringTok{"gray50"}\NormalTok{),}
                     \AttributeTok{name =} \StringTok{"Training Volume"}\NormalTok{) }\SpecialCharTok{+}  \CommentTok{\# \textless{}{-}{-} hide the color legend}
  
  \FunctionTok{scale\_fill\_manual}\NormalTok{(}\AttributeTok{values =} \FunctionTok{c}\NormalTok{(}\StringTok{"low"} \OtherTok{=} \StringTok{"\#E69F00"}\NormalTok{, }\StringTok{"mod"} \OtherTok{=} \StringTok{"\#56B4E9"}\NormalTok{, }\StringTok{"none"} \OtherTok{=} \StringTok{"gray50"}\NormalTok{),}
                    \AttributeTok{guide =} \StringTok{"none"}\NormalTok{) }\SpecialCharTok{+}  \CommentTok{\# \textless{}{-}{-} add fill scale}
  
  \FunctionTok{labs}\NormalTok{(}\AttributeTok{title =} \StringTok{""}\NormalTok{,}
       \AttributeTok{subtitle =} \StringTok{""}\NormalTok{,}
       \AttributeTok{x =} \StringTok{"Time"}\NormalTok{,}
       \AttributeTok{y =} \StringTok{"Peak Torque (Nm)"}\NormalTok{) }\SpecialCharTok{+}
  
  \FunctionTok{theme\_minimal}\NormalTok{() }\SpecialCharTok{+}
  \FunctionTok{theme}\NormalTok{(}\AttributeTok{legend.position =} \StringTok{"bottom"}\NormalTok{) }\SpecialCharTok{+}
  \FunctionTok{theme}\NormalTok{(}\AttributeTok{panel.grid.major =} \FunctionTok{element\_blank}\NormalTok{(),}
        \AttributeTok{panel.grid.minor =} \FunctionTok{element\_blank}\NormalTok{(),}
        \AttributeTok{axis.line =} \FunctionTok{element\_line}\NormalTok{(}\AttributeTok{color =} \StringTok{"black"}\NormalTok{))}


\CommentTok{\# 240 d/s}
\NormalTok{pti240mean }\OtherTok{\textless{}{-}}\NormalTok{ pt240 }\SpecialCharTok{\%\textgreater{}\%}
  \CommentTok{\#summarising means}
  \FunctionTok{group\_by}\NormalTok{(age\_group, allocation, condition, time) }\SpecialCharTok{\%\textgreater{}\%}
  \FunctionTok{summarise}\NormalTok{(}\AttributeTok{mean =} \FunctionTok{mean}\NormalTok{(value, }\AttributeTok{na.rm =} \ConstantTok{TRUE}\NormalTok{),}
            \AttributeTok{sd =} \FunctionTok{sd}\NormalTok{(value, }\AttributeTok{na.rm =} \ConstantTok{TRUE}\NormalTok{),}
            \AttributeTok{.groups =} \StringTok{"drop"}\NormalTok{ ) }\SpecialCharTok{\%\textgreater{}\%}
  
  \CommentTok{\#creating the plot}
  \FunctionTok{ggplot}\NormalTok{(}\FunctionTok{aes}\NormalTok{(}\AttributeTok{x =}\NormalTok{ time, }\AttributeTok{y =}\NormalTok{ mean, }\AttributeTok{color =}\NormalTok{ condition, }\AttributeTok{group =}\NormalTok{ condition)) }\SpecialCharTok{+}
  
  \FunctionTok{geom\_line}\NormalTok{(}\AttributeTok{linewidth =} \DecValTok{1}\NormalTok{, }\AttributeTok{position =} \FunctionTok{position\_dodge}\NormalTok{(.}\DecValTok{5}\NormalTok{)) }\SpecialCharTok{+}
  \FunctionTok{geom\_point}\NormalTok{(}\AttributeTok{shape =} \DecValTok{21}\NormalTok{, }\FunctionTok{aes}\NormalTok{(}\AttributeTok{fill =}\NormalTok{ condition), }\AttributeTok{color =} \StringTok{"black"}\NormalTok{,}
             \AttributeTok{size =} \DecValTok{3}\NormalTok{, }\AttributeTok{stroke =} \DecValTok{1}\NormalTok{, }\AttributeTok{position =} \FunctionTok{position\_dodge}\NormalTok{(.}\DecValTok{5}\NormalTok{)) }\SpecialCharTok{+}
  \CommentTok{\#geom\_errorbar(aes(ymin = mean {-} sd, ymax = mean + sd),}
   \CommentTok{\#             width = 0.1, linewidth = 0.8, position = position\_dodge(.2)) +}
  
  \CommentTok{\# adding in the raw data}
  \FunctionTok{geom\_point}\NormalTok{(}\AttributeTok{data =}\NormalTok{ pt240,}
             \FunctionTok{aes}\NormalTok{(}\AttributeTok{x =}\NormalTok{ time, }\AttributeTok{y =}\NormalTok{ value,}
                 \AttributeTok{color =}\NormalTok{ condition, }\AttributeTok{group =}\NormalTok{ condition),}
                 \AttributeTok{size =} \FloatTok{1.5}\NormalTok{,}
                 \AttributeTok{alpha =} \FloatTok{0.3}\NormalTok{,}
             \AttributeTok{position =} \FunctionTok{position\_jitter}\NormalTok{(}\AttributeTok{width =} \FloatTok{0.2}\NormalTok{)) }\SpecialCharTok{+}
  
  \FunctionTok{facet\_wrap}\NormalTok{(}\FunctionTok{vars}\NormalTok{(age\_group, allocation), }\AttributeTok{scales =} \StringTok{"fixed"}\NormalTok{) }\SpecialCharTok{+}
  
  \FunctionTok{scale\_color\_manual}\NormalTok{(}\AttributeTok{values =} \FunctionTok{c}\NormalTok{(}\StringTok{"low"} \OtherTok{=} \StringTok{"\#E69F00"}\NormalTok{, }\StringTok{"mod"} \OtherTok{=} \StringTok{"\#56B4E9"}\NormalTok{, }\StringTok{"none"} \OtherTok{=} \StringTok{"gray50"}\NormalTok{),}
                     \AttributeTok{name =} \StringTok{"Training Volume"}\NormalTok{) }\SpecialCharTok{+}  \CommentTok{\# \textless{}{-}{-} hide the color legend}
  
  \FunctionTok{scale\_fill\_manual}\NormalTok{(}\AttributeTok{values =} \FunctionTok{c}\NormalTok{(}\StringTok{"low"} \OtherTok{=} \StringTok{"\#E69F00"}\NormalTok{, }\StringTok{"mod"} \OtherTok{=} \StringTok{"\#56B4E9"}\NormalTok{, }\StringTok{"none"} \OtherTok{=} \StringTok{"gray50"}\NormalTok{),}
                    \AttributeTok{guide =} \StringTok{"none"}\NormalTok{) }\SpecialCharTok{+}  \CommentTok{\# \textless{}{-}{-} add fill scale}
  
  \FunctionTok{labs}\NormalTok{(}\AttributeTok{title =} \StringTok{""}\NormalTok{,}
       \AttributeTok{subtitle =} \StringTok{""}\NormalTok{,}
       \AttributeTok{x =} \StringTok{"Time"}\NormalTok{,}
       \AttributeTok{y =} \StringTok{""}\NormalTok{) }\SpecialCharTok{+}
  
  \FunctionTok{theme\_minimal}\NormalTok{() }\SpecialCharTok{+}
  \FunctionTok{theme}\NormalTok{(}\AttributeTok{legend.position =} \StringTok{"bottom"}\NormalTok{) }\SpecialCharTok{+}
  \FunctionTok{theme}\NormalTok{(}\AttributeTok{panel.grid.major =} \FunctionTok{element\_blank}\NormalTok{(),}
        \AttributeTok{panel.grid.minor =} \FunctionTok{element\_blank}\NormalTok{(),}
        \AttributeTok{axis.line =} \FunctionTok{element\_line}\NormalTok{(}\AttributeTok{color =} \StringTok{"black"}\NormalTok{))}



\CommentTok{\# Combinding the plots}

\NormalTok{ptimeanfig }\OtherTok{\textless{}{-}} \FunctionTok{plot\_grid}\NormalTok{(}
\NormalTok{  ptiisommean }\SpecialCharTok{+} \FunctionTok{theme}\NormalTok{(}\AttributeTok{legend.position =} \StringTok{"none"}\NormalTok{),}
\NormalTok{  pti60mean }\SpecialCharTok{+} \FunctionTok{theme}\NormalTok{(}\AttributeTok{legend.position =} \StringTok{"none"}\NormalTok{),}
\NormalTok{  pti120mean }\SpecialCharTok{+} \FunctionTok{theme}\NormalTok{(}\AttributeTok{legend.position =} \StringTok{"none"}\NormalTok{),}
\NormalTok{  pti240mean }\SpecialCharTok{+} \FunctionTok{theme}\NormalTok{(}\AttributeTok{legend.position =} \StringTok{"none"}\NormalTok{),}
  \AttributeTok{ncol =} \DecValTok{2}\NormalTok{,}
  \AttributeTok{nrow =} \DecValTok{2}\NormalTok{,}
  \AttributeTok{labels =} \FunctionTok{c}\NormalTok{(}\StringTok{"A) Isometric"}\NormalTok{, }\StringTok{"B) 60 d/s"}\NormalTok{, }\StringTok{"C) 120 d/s"}\NormalTok{, }\StringTok{"D) 240 d/s"}\NormalTok{),}
  \AttributeTok{label\_size =} \DecValTok{8}
\NormalTok{)}



\NormalTok{ptimeanfig}
\end{Highlighting}
\end{Shaded}

\pandocbounded{\includegraphics[keepaspectratio]{manuscript_files/figure-pdf/Plot3-1.pdf}}

\textbf{Plot 4: Mean Peak Torque change in a Hill curve format}

\begin{Shaded}
\begin{Highlighting}[]
\CommentTok{\# Calculating group means}
\NormalTok{curve.sum }\OtherTok{\textless{}{-}}\NormalTok{ long.dat }\SpecialCharTok{\%\textgreater{}\%}
  \FunctionTok{filter}\NormalTok{(outcome }\SpecialCharTok{==} \StringTok{"pt\_max"}\NormalTok{) }\SpecialCharTok{\%\textgreater{}\%}
  \FunctionTok{group\_by}\NormalTok{(age\_group, allocation, time, speed) }\SpecialCharTok{\%\textgreater{}\%}
  \FunctionTok{summarise}\NormalTok{(}\AttributeTok{mean\_torque =} \FunctionTok{mean}\NormalTok{(value, }\AttributeTok{na.rm =} \ConstantTok{TRUE}\NormalTok{),}
            \AttributeTok{se\_torque =} \FunctionTok{sd}\NormalTok{(value, }\AttributeTok{na.rm =} \ConstantTok{TRUE}\NormalTok{) }\SpecialCharTok{/} \FunctionTok{sqrt}\NormalTok{(}\FunctionTok{n}\NormalTok{()),}
            \AttributeTok{.groups =} \StringTok{"drop"}\NormalTok{)}

\CommentTok{\# Creating a combined group variable for easier plotting}
\NormalTok{curve.sum }\OtherTok{\textless{}{-}}\NormalTok{ curve.sum }\SpecialCharTok{\%\textgreater{}\%}
  \FunctionTok{mutate}\NormalTok{(}\AttributeTok{group =} \FunctionTok{paste}\NormalTok{(age\_group, allocation, }\AttributeTok{sep =} \StringTok{"\_"}\NormalTok{))}


\CommentTok{\# Plot 1: Comparing changes over time (baseline + halfway + post)}
\CommentTok{\# Separate panels for each group}

\NormalTok{plot\_time }\OtherTok{\textless{}{-}}\NormalTok{ curve.sum }\SpecialCharTok{\%\textgreater{}\%}
  \FunctionTok{ggplot}\NormalTok{(}\FunctionTok{aes}\NormalTok{(}\AttributeTok{x =}\NormalTok{ speed, }\AttributeTok{y =}\NormalTok{ mean\_torque, }\AttributeTok{color =}\NormalTok{ time, }\AttributeTok{group =}\NormalTok{ time)) }\SpecialCharTok{+}
  \FunctionTok{geom\_line}\NormalTok{(}\AttributeTok{linewidth =} \DecValTok{1}\NormalTok{) }\SpecialCharTok{+}
  \CommentTok{\#geom\_point(size = 3) +}
  \CommentTok{\#geom\_errorbar(aes(ymin = mean\_torque {-} se\_torque, }
  \CommentTok{\#                  ymax = mean\_torque + se\_torque), }
  \CommentTok{\#              width = 10) +}
  \FunctionTok{facet\_wrap}\NormalTok{(}\SpecialCharTok{\textasciitilde{}}\NormalTok{ age\_group }\SpecialCharTok{+}\NormalTok{ allocation, }\AttributeTok{ncol =} \DecValTok{3}\NormalTok{) }\SpecialCharTok{+}
  \FunctionTok{labs}\NormalTok{(}\AttributeTok{x =} \StringTok{"Angular Velocity (d/s)"}\NormalTok{,}
       \AttributeTok{y =} \StringTok{"Peak Torque (Nm)"}\NormalTok{,}
       \AttributeTok{color =} \StringTok{"Time Point"}\NormalTok{,}
       \AttributeTok{title =} \StringTok{"Force{-}Velocity Curves Across Training"}\NormalTok{) }\SpecialCharTok{+}
  \FunctionTok{theme\_classic}\NormalTok{() }\SpecialCharTok{+}
  \FunctionTok{theme}\NormalTok{(}\AttributeTok{legend.position =} \StringTok{"bottom"}\NormalTok{)}


\CommentTok{\# Plot 2: Single plot showing everything (might be busy)}
\NormalTok{curve.plot }\OtherTok{\textless{}{-}}\NormalTok{ curve.sum }\SpecialCharTok{\%\textgreater{}\%}
  \FunctionTok{ggplot}\NormalTok{(}\FunctionTok{aes}\NormalTok{(}\AttributeTok{x =}\NormalTok{ speed, }\AttributeTok{y =}\NormalTok{ mean\_torque, }\AttributeTok{color =}\NormalTok{ group, }\AttributeTok{linetype =}\NormalTok{ time, }
             \AttributeTok{group =} \FunctionTok{interaction}\NormalTok{(group, time))) }\SpecialCharTok{+}
  \FunctionTok{geom\_line}\NormalTok{(}\AttributeTok{linewidth =} \DecValTok{1}\NormalTok{) }\SpecialCharTok{+}
  \FunctionTok{geom\_point}\NormalTok{(}\AttributeTok{size =} \DecValTok{2}\NormalTok{) }\SpecialCharTok{+}
  \FunctionTok{labs}\NormalTok{(}\AttributeTok{x =} \StringTok{"Angular Velocity (d/s)"}\NormalTok{,}
       \AttributeTok{y =} \StringTok{"Peak Torque (Nm)"}\NormalTok{,}
       \AttributeTok{color =} \StringTok{"Group"}\NormalTok{,}
       \AttributeTok{linetype =} \StringTok{"Time Point"}\NormalTok{,}
       \AttributeTok{title =} \StringTok{""}\NormalTok{) }\SpecialCharTok{+}
  \FunctionTok{scale\_color\_manual}\NormalTok{(}\AttributeTok{values =} \FunctionTok{c}\NormalTok{(}\StringTok{"yng\_int"} \OtherTok{=} \StringTok{"\#E69F00"}\NormalTok{,}
                                  \StringTok{"old\_int"} \OtherTok{=} \StringTok{"\#56B4E9"}\NormalTok{,}
                                  \StringTok{"old\_con"} \OtherTok{=} \StringTok{"gray50"}\NormalTok{),}
                     \AttributeTok{labels =} \FunctionTok{c}\NormalTok{(}\StringTok{"yng\_int"} \OtherTok{=} \StringTok{"Young Intervention"}\NormalTok{, }
                                \StringTok{"old\_int"} \OtherTok{=} \StringTok{"Old Intervention"}\NormalTok{, }
                                \StringTok{"old\_con"} \OtherTok{=} \StringTok{"Old Control"}\NormalTok{)) }\SpecialCharTok{+}
  \FunctionTok{theme\_classic}\NormalTok{() }\SpecialCharTok{+}
  \FunctionTok{theme}\NormalTok{(}\AttributeTok{legend.position =} \StringTok{"bottom"}\NormalTok{)}


\NormalTok{curve.plot}
\end{Highlighting}
\end{Shaded}

\pandocbounded{\includegraphics[keepaspectratio]{manuscript_files/figure-pdf/Plot4-1.pdf}}

\section*{Bibliography}\label{bibliography}
\addcontentsline{toc}{section}{Bibliography}

\phantomsection\label{refs}
\begin{CSLReferences}{1}{0}
\bibitem[\citeproctext]{ref-digitale_tutorial_2022}
Digitale, J. C., Martin, J. N., \& Glymour, M. M. M. (2022). Tutorial on
directed acyclic graphs. \emph{Journal of Clinical Epidemiology},
\emph{142}, 264--267.
\url{https://doi.org/10.1016/j.jclinepi.2021.08.001}

\bibitem[\citeproctext]{ref-Wilde_2000}
Dw, W., Kd, M., Gk, W., A, V., \& Rj, G. (2000). High-fat diet elevates
blood pressure and cerebrovascular muscle ca(2+) current.
\emph{Hypertension (Dallas, Tex. : 1979)}, \emph{35}(3).
\url{https://doi.org/10.1161/01.hyp.35.3.832}

\bibitem[\citeproctext]{ref-Miall_1967}
Miall, W. E., \& Lovell, H. G. (1967). Relation between change of blood
pressure and age. \emph{British Medical Journal}, \emph{2}(5553), 660.
\url{https://doi.org/10.1136/bmj.2.5553.660}

\bibitem[\citeproctext]{ref-McElreath_2018}
\emph{Statistical rethinking : A bayesian course with examples in r and
stan}. (2018). Chapman; Hall/{CRC}.
\url{https://doi.org/10.1201/9781315372495}

\end{CSLReferences}




\end{document}

% Options for packages loaded elsewhere
% Options for packages loaded elsewhere
\PassOptionsToPackage{unicode}{hyperref}
\PassOptionsToPackage{hyphens}{url}
\PassOptionsToPackage{dvipsnames,svgnames,x11names}{xcolor}
%
\documentclass[
  letterpaper,
  DIV=11,
  numbers=noendperiod]{scrartcl}
\usepackage{xcolor}
\usepackage{amsmath,amssymb}
\setcounter{secnumdepth}{-\maxdimen} % remove section numbering
\usepackage{iftex}
\ifPDFTeX
  \usepackage[T1]{fontenc}
  \usepackage[utf8]{inputenc}
  \usepackage{textcomp} % provide euro and other symbols
\else % if luatex or xetex
  \usepackage{unicode-math} % this also loads fontspec
  \defaultfontfeatures{Scale=MatchLowercase}
  \defaultfontfeatures[\rmfamily]{Ligatures=TeX,Scale=1}
\fi
\usepackage{lmodern}
\ifPDFTeX\else
  % xetex/luatex font selection
\fi
% Use upquote if available, for straight quotes in verbatim environments
\IfFileExists{upquote.sty}{\usepackage{upquote}}{}
\IfFileExists{microtype.sty}{% use microtype if available
  \usepackage[]{microtype}
  \UseMicrotypeSet[protrusion]{basicmath} % disable protrusion for tt fonts
}{}
\makeatletter
\@ifundefined{KOMAClassName}{% if non-KOMA class
  \IfFileExists{parskip.sty}{%
    \usepackage{parskip}
  }{% else
    \setlength{\parindent}{0pt}
    \setlength{\parskip}{6pt plus 2pt minus 1pt}}
}{% if KOMA class
  \KOMAoptions{parskip=half}}
\makeatother
% Make \paragraph and \subparagraph free-standing
\makeatletter
\ifx\paragraph\undefined\else
  \let\oldparagraph\paragraph
  \renewcommand{\paragraph}{
    \@ifstar
      \xxxParagraphStar
      \xxxParagraphNoStar
  }
  \newcommand{\xxxParagraphStar}[1]{\oldparagraph*{#1}\mbox{}}
  \newcommand{\xxxParagraphNoStar}[1]{\oldparagraph{#1}\mbox{}}
\fi
\ifx\subparagraph\undefined\else
  \let\oldsubparagraph\subparagraph
  \renewcommand{\subparagraph}{
    \@ifstar
      \xxxSubParagraphStar
      \xxxSubParagraphNoStar
  }
  \newcommand{\xxxSubParagraphStar}[1]{\oldsubparagraph*{#1}\mbox{}}
  \newcommand{\xxxSubParagraphNoStar}[1]{\oldsubparagraph{#1}\mbox{}}
\fi
\makeatother

\usepackage{color}
\usepackage{fancyvrb}
\newcommand{\VerbBar}{|}
\newcommand{\VERB}{\Verb[commandchars=\\\{\}]}
\DefineVerbatimEnvironment{Highlighting}{Verbatim}{commandchars=\\\{\}}
% Add ',fontsize=\small' for more characters per line
\usepackage{framed}
\definecolor{shadecolor}{RGB}{241,243,245}
\newenvironment{Shaded}{\begin{snugshade}}{\end{snugshade}}
\newcommand{\AlertTok}[1]{\textcolor[rgb]{0.68,0.00,0.00}{#1}}
\newcommand{\AnnotationTok}[1]{\textcolor[rgb]{0.37,0.37,0.37}{#1}}
\newcommand{\AttributeTok}[1]{\textcolor[rgb]{0.40,0.45,0.13}{#1}}
\newcommand{\BaseNTok}[1]{\textcolor[rgb]{0.68,0.00,0.00}{#1}}
\newcommand{\BuiltInTok}[1]{\textcolor[rgb]{0.00,0.23,0.31}{#1}}
\newcommand{\CharTok}[1]{\textcolor[rgb]{0.13,0.47,0.30}{#1}}
\newcommand{\CommentTok}[1]{\textcolor[rgb]{0.37,0.37,0.37}{#1}}
\newcommand{\CommentVarTok}[1]{\textcolor[rgb]{0.37,0.37,0.37}{\textit{#1}}}
\newcommand{\ConstantTok}[1]{\textcolor[rgb]{0.56,0.35,0.01}{#1}}
\newcommand{\ControlFlowTok}[1]{\textcolor[rgb]{0.00,0.23,0.31}{\textbf{#1}}}
\newcommand{\DataTypeTok}[1]{\textcolor[rgb]{0.68,0.00,0.00}{#1}}
\newcommand{\DecValTok}[1]{\textcolor[rgb]{0.68,0.00,0.00}{#1}}
\newcommand{\DocumentationTok}[1]{\textcolor[rgb]{0.37,0.37,0.37}{\textit{#1}}}
\newcommand{\ErrorTok}[1]{\textcolor[rgb]{0.68,0.00,0.00}{#1}}
\newcommand{\ExtensionTok}[1]{\textcolor[rgb]{0.00,0.23,0.31}{#1}}
\newcommand{\FloatTok}[1]{\textcolor[rgb]{0.68,0.00,0.00}{#1}}
\newcommand{\FunctionTok}[1]{\textcolor[rgb]{0.28,0.35,0.67}{#1}}
\newcommand{\ImportTok}[1]{\textcolor[rgb]{0.00,0.46,0.62}{#1}}
\newcommand{\InformationTok}[1]{\textcolor[rgb]{0.37,0.37,0.37}{#1}}
\newcommand{\KeywordTok}[1]{\textcolor[rgb]{0.00,0.23,0.31}{\textbf{#1}}}
\newcommand{\NormalTok}[1]{\textcolor[rgb]{0.00,0.23,0.31}{#1}}
\newcommand{\OperatorTok}[1]{\textcolor[rgb]{0.37,0.37,0.37}{#1}}
\newcommand{\OtherTok}[1]{\textcolor[rgb]{0.00,0.23,0.31}{#1}}
\newcommand{\PreprocessorTok}[1]{\textcolor[rgb]{0.68,0.00,0.00}{#1}}
\newcommand{\RegionMarkerTok}[1]{\textcolor[rgb]{0.00,0.23,0.31}{#1}}
\newcommand{\SpecialCharTok}[1]{\textcolor[rgb]{0.37,0.37,0.37}{#1}}
\newcommand{\SpecialStringTok}[1]{\textcolor[rgb]{0.13,0.47,0.30}{#1}}
\newcommand{\StringTok}[1]{\textcolor[rgb]{0.13,0.47,0.30}{#1}}
\newcommand{\VariableTok}[1]{\textcolor[rgb]{0.07,0.07,0.07}{#1}}
\newcommand{\VerbatimStringTok}[1]{\textcolor[rgb]{0.13,0.47,0.30}{#1}}
\newcommand{\WarningTok}[1]{\textcolor[rgb]{0.37,0.37,0.37}{\textit{#1}}}

\usepackage{longtable,booktabs,array}
\usepackage{calc} % for calculating minipage widths
% Correct order of tables after \paragraph or \subparagraph
\usepackage{etoolbox}
\makeatletter
\patchcmd\longtable{\par}{\if@noskipsec\mbox{}\fi\par}{}{}
\makeatother
% Allow footnotes in longtable head/foot
\IfFileExists{footnotehyper.sty}{\usepackage{footnotehyper}}{\usepackage{footnote}}
\makesavenoteenv{longtable}
\usepackage{graphicx}
\makeatletter
\newsavebox\pandoc@box
\newcommand*\pandocbounded[1]{% scales image to fit in text height/width
  \sbox\pandoc@box{#1}%
  \Gscale@div\@tempa{\textheight}{\dimexpr\ht\pandoc@box+\dp\pandoc@box\relax}%
  \Gscale@div\@tempb{\linewidth}{\wd\pandoc@box}%
  \ifdim\@tempb\p@<\@tempa\p@\let\@tempa\@tempb\fi% select the smaller of both
  \ifdim\@tempa\p@<\p@\scalebox{\@tempa}{\usebox\pandoc@box}%
  \else\usebox{\pandoc@box}%
  \fi%
}
% Set default figure placement to htbp
\def\fps@figure{htbp}
\makeatother





\setlength{\emergencystretch}{3em} % prevent overfull lines

\providecommand{\tightlist}{%
  \setlength{\itemsep}{0pt}\setlength{\parskip}{0pt}}



 


\KOMAoption{captions}{tableheading}
\usepackage{wrapfig}
\makeatletter
\@ifpackageloaded{caption}{}{\usepackage{caption}}
\AtBeginDocument{%
\ifdefined\contentsname
  \renewcommand*\contentsname{Table of contents}
\else
  \newcommand\contentsname{Table of contents}
\fi
\ifdefined\listfigurename
  \renewcommand*\listfigurename{List of Figures}
\else
  \newcommand\listfigurename{List of Figures}
\fi
\ifdefined\listtablename
  \renewcommand*\listtablename{List of Tables}
\else
  \newcommand\listtablename{List of Tables}
\fi
\ifdefined\figurename
  \renewcommand*\figurename{Figure}
\else
  \newcommand\figurename{Figure}
\fi
\ifdefined\tablename
  \renewcommand*\tablename{Table}
\else
  \newcommand\tablename{Table}
\fi
}
\@ifpackageloaded{float}{}{\usepackage{float}}
\floatstyle{ruled}
\@ifundefined{c@chapter}{\newfloat{codelisting}{h}{lop}}{\newfloat{codelisting}{h}{lop}[chapter]}
\floatname{codelisting}{Listing}
\newcommand*\listoflistings{\listof{codelisting}{List of Listings}}
\makeatother
\makeatletter
\makeatother
\makeatletter
\@ifpackageloaded{caption}{}{\usepackage{caption}}
\@ifpackageloaded{subcaption}{}{\usepackage{subcaption}}
\makeatother
\usepackage{bookmark}
\IfFileExists{xurl.sty}{\usepackage{xurl}}{} % add URL line breaks if available
\urlstyle{same}
\hypersetup{
  pdftitle={Manuscript},
  colorlinks=true,
  linkcolor={blue},
  filecolor={Maroon},
  citecolor={Blue},
  urlcolor={Blue},
  pdfcreator={LaTeX via pandoc}}


\title{Manuscript}
\author{}
\date{}
\begin{document}
\maketitle


\section{1. Introduction}\label{introduction}

\section{2. Setting Up a Collaborative
Workspace}\label{setting-up-a-collaborative-workspace}

\begin{itemize}
\tightlist
\item
  How to create a repository in Git Hub
\item
  How to link it to Rstudio
\item
  How to create a project in your local computer and export it to GitHub
\item
  Basic commands for Git and GitHub fromt he RStudio terminal
\end{itemize}

A collaborative workflow requires a collaborative work space that
enables everyone participating to share and contribute to a project.\\
There are multiple options for such a workspace, like Teams, Discord, or
even over e-mail. When working with code however, online repositories on
GitHub can be a good alternative.

In this tutorial we assume that you will be working with RStudio and
have already downloaded R and Rstudio. In addition you are going to need
to have the version control software Git installed and have an account
in GitHub. If you need guidance for this you can find a helpful tutorial
\href{1\%20Setting\%20up\%20your\%20software\%20environment\%20–\%20A\%20Crash\%20Course\%20in\%20R}{here}.

\subsubsection{What is a repository?}\label{what-is-a-repository}

A repository is basically like a project box where you collect all the
files, data, graphs and code scripts from your project.\\
Online repositories can be accessed from the internet and from any
computer, while a local repository is only stored in a specific computer
and cannot be accessed elsewhere. When setting up a collaborative work
space its advantageous to have an online repository so that multiple
people can contribute from their own computer to a shared repository,
without having to send files by mail etc. In addition we can connect the
online repository what a local repository which allows us to work and
make changes using our own computer and then we can upload it to the
online repository.

\subsubsection{What is Git?}\label{what-is-git}

Git is a version control software that allows you to track the different
version of your files. It basically allows you to keep a detailed
history of changes you have done in your document and also what other
people have added or removed in your collaborative documents. Having a
version controls software set up for your workflow is very handy as it
prevents major losses of documents and changes, and if any error is
introduced in a document or code, you can track it back to see what and
who submitted it. This fosters reproducibility, transparency,
collaboration and robustness for your project.

\subsubsection{What is GitHub?}\label{what-is-github}

GitHub is a collaborative online platform that allows you to host and
join online repositories. its kinda like facebook for coding. gitHub
allows us to share and collaborate with the people on the same code at
the same time. It can also be used to host webpages and other stuff.

In this toturial we will only work with the RStudio interface and the
online GitHub interface. however, if you want an expanded commandline
and interface for GitHub you can use GitHUb CLI and/or GitHub Desktop.
See toturials here:
\href{GitHub\%20CLI\%20quickstart\%20-\%20GitHub\%20Docs}{GitHub CLI}
and
\href{Getting\%20started\%20with\%20GitHub\%20Desktop\%20-\%20GitHub\%20Docs}{GitHub
Desktop}.

\subsection{2.1 How to create a project that is connected between
RStudio and
GitHub?}\label{how-to-create-a-project-that-is-connected-between-rstudio-and-github}

When creating a new project and you want to link your local project with
an online repository, you can go about it to ways basicly.\\
a) You can create the online repository and then clone it down to you
computer\\
b) You can create a local repository and then push it online to GitHub

We´ll go through both options here, starting with the online repository.

Image file path:

resources/images/

\newpage

\subsubsection{2.1.1 Starting with an online
repository}\label{starting-with-an-online-repository}

\paragraph{\texorpdfstring{Step 1. Create a new online repository in
GitHub
(\href{Creating\%20Your\%20First\%20GitHub\%20Repository\%20and\%20Pushing\%20Code\%20-\%20YouTube}{Video
tutorial})}{Step 1. Create a new online repository in GitHub (Video tutorial)}}\label{step-1.-create-a-new-online-repository-in-github-video-tutorial}

\begin{wrapfigure}{lt}{0.5\textwidth}
  \centering
  \includegraphics[width=0.38\textwidth]{resources/images/image_1.png}
  \caption{Creating a new online repository.}
  \vspace{-1.5cm} % Adjust vertical spacing below the figure
\end{wrapfigure}

Once you´ve logged into GitHub, navigate to the top right corner of your
page and find the + tab. Drop it down to reveal the ``New repository''
option. Click on it.

This will take you ta the repository creation page.\\
Here you give your repository a name, a description of what it will
entail and wherever it is public or not.

You also have the options of adding a README file and a .gitignore file
upon creation, but it is possible to create these after the repository
is made as well.

\subparagraph{README}\label{readme}

A README file is a descriptive file that should explain what the
project/repository is about, how it is organized and what the data in it
means etc. Any additional information you want people to know when using
your repository should go into the README.

\begin{wrapfigure}{r}{0.4\textwidth}
  \vspace{-1cm} % Adjust vertical spacing to remove extra white space
  \centering
  \includegraphics[width=0.50\textwidth]{resources/images/image_3.png}
  \caption{Repository setup-page.}
  \vspace{-2cm} % Adjust vertical spacing to remove extra white space
\end{wrapfigure}

\subparagraph{.gitignore}\label{gitignore}

The .gitignore file is an information file that tells Git what types of
files it should track, or specifically not track. This is useful when
you for example don't want to track the generated images or graphs from
your code, but just your code.

So, now that you have created your first online repository you want to
connect it to your local computer.\\
You can do this multiple ways, but in this tutorial we´ll use commands
in the terminal to initialize

\paragraph{\texorpdfstring{\textbf{Step 2: Copy the Repository
URL}}{Step 2: Copy the Repository URL}}\label{step-2-copy-the-repository-url}

\begin{enumerate}
\def\labelenumi{\arabic{enumi}.}
\item
  Go to your GitHub repository page.
\item
  Click the green \textbf{Code} button.
\item
  Copy the repository URL:

  \begin{itemize}
  \tightlist
  \item
    For example:
    \textbf{\texttt{https://github.com/yourusername/repositoryname.git}}
  \end{itemize}
\end{enumerate}

\paragraph{\texorpdfstring{\textbf{Step 3: Open RStudio and Clone the
Repository}}{Step 3: Open RStudio and Clone the Repository}}\label{step-3-open-rstudio-and-clone-the-repository}

\begin{enumerate}
\def\labelenumi{\arabic{enumi}.}
\item
  Open RStudio.
\item
  Go to \textbf{File} → \textbf{New Project} → \textbf{Version Control}
  → \textbf{Git}.
\item
  Paste the GitHub repository URL into the ``Repository URL'' field.
\item
  Choose a folder on your local computer where you want to clone the
  repository.
\item
  Click \textbf{Create Project}.when doing a commit on a file that has
  been staged, that version of the file goes into the version history.
  It is also tracked.
\end{enumerate}

And tada! You have now cloned the online repository to your computer!
Great job!

\paragraph{Step 4 (Optional): Configure Git in
RStudio}\label{step-4-optional-configure-git-in-rstudio}

If this is your first time using Git with RStudio, you'll need to
configure your Git credentials. Open the RStudio terminal
(\textbf{Tools}~→~\textbf{Terminal}) or navigate to the \textbf{terminal
tab} at the top right of the RStudio interface and run the following
commands:

\begin{Shaded}
\begin{Highlighting}[]
\FunctionTok{git}\NormalTok{ config –global user.name }\StringTok{"Your Name"}

\FunctionTok{git}\NormalTok{ config }\AttributeTok{{-}{-}global}\NormalTok{ user.email }\StringTok{"your\_email@example.com"}
\end{Highlighting}
\end{Shaded}

Replace the placeholder names in ``Your Name'' and
``your\_email@example.com'' with your own.

Also, if Git has not yet been initiated in your RStudio project you can
use the command:

\begin{Shaded}
\begin{Highlighting}[]
\FunctionTok{git}\NormalTok{ init}
\end{Highlighting}
\end{Shaded}

To check whet ever Git is initialized in your project you can write:

\begin{Shaded}
\begin{Highlighting}[]
\FunctionTok{git}\NormalTok{ status}
\end{Highlighting}
\end{Shaded}

If git is not initialized an error message will show up. Then just run
the \texttt{git\ init} command and follow the instructions.

\subsubsection{2.1.2 Starting with a local project in
RStudio}\label{starting-with-a-local-project-in-rstudio}

Now, what if you wanted to do it the other way around, like if you
already have a local project on your computer and want to create an
online repository for it?

\paragraph{\texorpdfstring{\textbf{Step 1. Create a New Local Project in
RStudio}}{Step 1. Create a New Local Project in RStudio}}\label{step-1.-create-a-new-local-project-in-rstudio}

If you already have a local project, you can skip this step. If not:

\begin{enumerate}
\def\labelenumi{\arabic{enumi}.}
\item
  Open \textbf{RStudio}.
\item
  Go to \textbf{File \textgreater{} New Project \textgreater{} New
  Directory \textgreater{} New Project}.
\item
  Choose a folder where you want the project to live and give it a name.
\item
  Make sure to check the box \textbf{Create a Git repository}.
\item
  Click \textbf{Create Project}.
\end{enumerate}

This initializes a local Git repository in your project directory.

If you already have made a project but it is not connected to a Git
repository you can do it like this:

\begin{enumerate}
\def\labelenumi{\arabic{enumi}.}
\tightlist
\item
  Navigate to \textbf{Tools \textgreater{} Project Options
  \textgreater{} Git/SVN}.
\item
  Select \textbf{Git} and click \textbf{Yes} when prompted to initialize
  a Git repository for your project.
\end{enumerate}

\paragraph{Step 2. Create a New Repository on
GitHub}\label{step-2.-create-a-new-repository-on-github}

\begin{enumerate}
\def\labelenumi{\arabic{enumi}.}
\item
  Create a new repository in Git Hub like previously in section 2.1.1
  step 1

  \begin{itemize}
  \tightlist
  \item
    Do \textbf{not} initialize the repository with a README,
    \textbf{\texttt{.gitignore}}, or license (we'll connect the existing
    local repository later).
  \end{itemize}
\end{enumerate}

You now have a new, empty GitHub repository.

\paragraph{Step 3. Link the Local project to the GitHub
Repository}\label{step-3.-link-the-local-project-to-the-github-repository}

\begin{enumerate}
\def\labelenumi{\arabic{enumi}.}
\tightlist
\item
  Copy the URL in the same way as previously in section 2.1.1 step 2.
\item
  Open the \textbf{Terminal} in RStudio (or use any terminal on your
  computer).
\item
  Navigate to your project folder, if you're not already there, by
  clicking on the dropdown menu in the top right corner of the RStudio
  interface and choose your project.
\item
  Add the GitHub repository as the remote origin using this command in
  the terminal:
\end{enumerate}

\begin{Shaded}
\begin{Highlighting}[]
\FunctionTok{git}\NormalTok{ remote add origin https://github.com/yourusername/your{-}repository.git}
\end{Highlighting}
\end{Shaded}

\begin{enumerate}
\def\labelenumi{\arabic{enumi}.}
\setcounter{enumi}{4}
\tightlist
\item
  Verify the remote connection using this command in the terminal
  \texttt{git\ remote\ -v}
\end{enumerate}

You should be able to see something like this:

\begin{Shaded}
\begin{Highlighting}[]
\ExtensionTok{origin}\NormalTok{  https://github.com/yourusername/your{-}repo.git }\ErrorTok{(}\ExtensionTok{fetch}\KeywordTok{)}
\ExtensionTok{origin}\NormalTok{  https://github.com/yourusername/your{-}repo.git }\ErrorTok{(}\ExtensionTok{push}\KeywordTok{)}
\end{Highlighting}
\end{Shaded}

That means you have now successfully established a connection between
your local project and the online repository on GitHub. Congratulations!

\subsection{2.2 Workflow between local and online
repositories}\label{workflow-between-local-and-online-repositories}

Now that we have connected our local repository with the online one, we
can start to pushing some code! But before we jump into the commands for
transferring files between our local and online repository, we should
better understand how these processes work.

Take a look at the figure below:

\begin{figure}[H]

{\centering \includegraphics[width=1\linewidth,height=0.6\textheight]{resources/images/Local_vs_online_workflow.png}

}

\caption{Visaluzation of local and online workflow in Rstudio.}

\end{figure}%

Your \textbf{working tree} (also called \textbf{working directory}) is
the project folder you are currently working in on your local computer.
This is where you do all your edits on your files and code.

The \textbf{index/staging area} is a list is a list of files that Git is
tracking for changes. When you ``stage'' files you tell Git to include
these files in the next commit.

The \textbf{local branch} is a version of your project that exists
entirely on your local computer. When you ``commit'' changes, you create
a permanent snapshot of the files currently staged in the index and save
them to your local repository.

The \textbf{remote tracking ref} is where you track the state of the
online repository. When you ``fetch'' changes from the online
repository, Git downloads the latest updates from the online repository
but without merging it into your working directory. This allows you to
see changes from collaborators or updates from the remote repository
while keeping your working tree unaffected until you explicitly merge or
rebase the changes.

\paragraph{2.2.1 Commands}\label{commands}

Okay, now we can take a look at the commands we can use for this
workflow.

Lets say you have edited some files in your working directory and want
to add them to your~\textbf{staging area}~before committing them to your
repository.~ If you only want to ``stage'' some files, but not all, you
can use the command in the terminal:

\begin{Shaded}
\begin{Highlighting}[]
\FunctionTok{git}\NormalTok{ add file\_name.txt}
\end{Highlighting}
\end{Shaded}

Just replace the file\_name.txt with the name of the file that you want
to stage. Remember to also include the file extentions (like .txt or
.pdf etc.)

If you want to stage \textbf{all files in your repository} that have had
changes to them you can write:

\begin{Shaded}
\begin{Highlighting}[]
\FunctionTok{git}\NormalTok{ add }\AttributeTok{{-}A}
\end{Highlighting}
\end{Shaded}

This command stages \textbf{all changes} in the repository, including
modified files, newly created files and deleted files.

Great! Now your edited files are ready to be committed! You can commit
your files using this command:

\begin{Shaded}
\begin{Highlighting}[]
\FunctionTok{git}\NormalTok{ commit }\AttributeTok{{-}m} \StringTok{"Commit message"}
\end{Highlighting}
\end{Shaded}

When you run this commit command, all the files that you have staged are
committed to your \textbf{local branch!} Its good to include a
\textbf{descriptive commit message} that explains what your commit
entails. That way its easier to track where changes or errors are
introduced in your repository. Messages could be ``Changed font
headings'' or ``Added statistical ANOVA analysis to data processing''.

Now that you have committed the files you want to push them to your
remote directory using:

\begin{Shaded}
\begin{Highlighting}[]
\FunctionTok{git}\NormalTok{ push}
\end{Highlighting}
\end{Shaded}

This command uploads everything you have committed so far this session
to your online repository and updates it based on your commits. Using
this command you can push multiple commits at the same time.

Now lets say your colleague, who you are collaborate with, has added a
new file to the online repository on GitHub. You want to pull that
document from the online directory down to your local repositopry and
start editing it. The most straightforward way is to use:

\begin{Shaded}
\begin{Highlighting}[]
\FunctionTok{git}\NormalTok{ pull}
\end{Highlighting}
\end{Shaded}

This command pulls the latest changes from your remote repository and
merges them into your~\textbf{current branch}~in your local repository,
so that they are exactly the same. Then you can just start editing and
making changes.

If you don't want to fully copy the online repository yet, but take a
look at it first before merging it ito your working directory you can
use the command

\begin{Shaded}
\begin{Highlighting}[]
\FunctionTok{git}\NormalTok{ fetch}
\end{Highlighting}
\end{Shaded}

This command downloads the changes from the remote repository
(e.g.,~\textbf{\texttt{origin}}) and updates your remote-tracking
branches (e.g.,~\textbf{\texttt{origin/main}}) without modifying your
working directory or local branch.

To look at the branch you have fetched you can use one of these two
commands:

\begin{Shaded}
\begin{Highlighting}[]
\FunctionTok{git}\NormalTok{ log HEAD..origin/main}

\FunctionTok{git}\NormalTok{ diff HEAD..origin/main}
\end{Highlighting}
\end{Shaded}

\textbf{git log} will give you a list of the commits that are not in
your local branch (\textbf{HEAD}) together with their metadata (authors,
date etc.). This is helpful if you want to quickly answer ``What was
changed?'' and ``Who changed it?''.

\textbf{git diff} will show you the actual changes that are between two
points, ex. your local branch (\textbf{HEAD}) and your fetched branch
(\textbf{origin/main}). git diff will show you the specific lines of
code that were added modified or deleted in the fetched branch. This is
helpful if you want to know exactly what was changed.

If you then want to integrate the fetched branch with your current
branch, you can use the command:

\begin{Shaded}
\begin{Highlighting}[]
\FunctionTok{git}\NormalTok{ merge}
\end{Highlighting}
\end{Shaded}

Using this command, Git creates a \textbf{merge commit}, which combines
the changes from both branches while preserving the complete commit
history of both branches. The merge commit explicitly shows the point
where the branches diverged and were brought together.

An alternative way to integrate the fetched branch is to use:

\begin{Shaded}
\begin{Highlighting}[]
\FunctionTok{git}\NormalTok{ rebase}
\end{Highlighting}
\end{Shaded}

This command does~\textbf{not create a merge commit}. Instead,
it~\textbf{rewrites the commit history}~of your current branch by
replaying its commits on top of the fetched branch, creating
a~\textbf{linear history}. Essentially, it re-aligns your current branch
so that it starts from the latest commit of the fetched branch, as if
your changes were made after the fetched branch's changes. This results
in a cleaner, more linear commit history.

Questions:

\begin{itemize}
\item
  add sections about README files and .gitignore
\item
  should I add a section about the concole and the terminal andhow they
  are used? Or could this perhaps go into the Introduction?
\item
  Should we include stuff about creating branches?
\end{itemize}

\textbf{Disclaimer:} This section was written with the help of SIKT KI
chat using the gpt-4o model. The AI was used to verify code chunks,
summarize steps in an organized format and rewrite original text for
better grammar and flow. The author takes full responsibility for the
resulting output. \href{Sikt\%20KI}{Link to AI model}.

\newpage

\section{3. Conducting Simulations Before Data
Acquisition}\label{conducting-simulations-before-data-acquisition}

\section{4. Including Data Packages for
Distribution}\label{including-data-packages-for-distribution}

\section{5. Creating Visualizations from Data
Packages}\label{creating-visualizations-from-data-packages}




\end{document}
